
\documentclass[DIV=calc]{scrartcl}
\usepackage[utf8]{inputenc}
\usepackage[T1]{fontenc}
\usepackage[ngerman]{babel}
\usepackage{graphicx}

\usepackage{ulem}
%\usepackage[dvipsnames]{xcolor}
\usepackage{paralist}
\usepackage{fixltx2e}
%\usepackage{ellipsis}
\usepackage[tracking=true]{microtype}

\usepackage{lmodern}                        % Ersatz fuer Computer Modern-Schriften
%\usepackage{hfoldsty}

%\usepackage{fourier}             % Schriftart
\usepackage[scaled=0.81]{helvet}     % Schriftart

\usepackage{url}
\usepackage{tocloft}             % Paket für Table of Contents

\usepackage{xcolor}
\definecolor{urlred}{HTML}{660000}

\usepackage{hyperref}
\hypersetup{
    colorlinks=true,    
    linkcolor=black,    % Farbe der internen Links (u.a. Table of Contents)
    urlcolor=black,    % Farbe der url-links
    citecolor=black} % Farbe der Literaturverzeichnis-Links

\usepackage{mdwlist}     % Änderung der Zeilenabstände bei itemize und enumerate
\usepackage{draftwatermark} % Wasserzeichen ``Entwurf'' 
\SetWatermarkText{}%Entwurf}

\usepackage{blindtext}
\parindent 0pt                 % Absatzeinrücken verhindern
\parskip 12pt                 % Absätze durch Lücke trennen

%\usepackage{titlesec}    % Abstand nach Überschriften neu definieren
%\titlespacing{\subsection}{0ex}{3ex}{-1ex}
%\titlespacing{\subsubsection}{0ex}{3ex}{-1ex}        

% \pagestyle{empty}
\setlength{\textheight}{23cm}
\usepackage{fancyhdr}
\pagestyle{fancy}
\cfoot{}
\lfoot{Zusammenkunft aller Physik-Fachschaften}
\rfoot{www.zapfev.de\\stapf@zapf.in}
\renewcommand{\headrulewidth}{0pt}
\renewcommand{\footrulewidth}{0.1pt}
\newcommand{\gen}{*innen}

\begin{document}
    \hspace{0.87\textwidth}
    \begin{minipage}{120pt}
        \vspace{-1.8cm}
        \includegraphics[width=80pt]{logo.pdf}
        \centering
        \small Zusammenkunft aller Physik-Fachschaften
    \end{minipage}
    \begin{center}
        \huge{Positionspapier der Zusammenkunft aller Physik-Fachschaften}\vspace{.25\baselineskip}\\
        \normalsize
    \end{center}
    \vspace{1cm} 
    \section*{Gegen kommerzielle Werbung in Lern- und Lehrräumen}
Die Zusammenkunft aller Physik Fachschaften (ZaPF)  spricht sich dafür aus, dass in Räumen der Lehre und des Lernens (z.B. Bibliotheken, Hörsäle, Übungsräume, Praktikumsräume) bei Lehr- und Lernbetrieb das Arbeiten ohne Beeinflussung durch Werbung stattfinden soll. Sinn der Lehrveranstaltungen und des Lernbetriebs ist es, dass Studierende unbeeinflusst von Interessen Dritter Fachinhalte erlernen und diskutieren, sowie Lehrende Lehrinhalte frei vermitteln können. Diese Arbeitsatmosphäre wird durch Werbung beeinträchtigt. 
Kommerzielle Werbung [1]  in diesen Räumen, insbesondere Hörsaal- und Raumbranding [2] ist daher nicht hinnehmbar. 


[1]: Werbung meint hier Maßnahmen zur Öffentlichkeitswirkung von kommerziellen, außeruniversitären Einrichtungen.

[2]: Hörsaal- und Raumbranding meint hier den Verkauf von Namensrechten von Hörsälen und anderen Lehr- und Lernräumen. In konkreten Fällen kann dies das Anbringen von Firmenlogos am und im betroffenen Raum und an der Rauminfrastruktur, sowie die Eintragung des Namens ins Raumverwaltungssystem der Hochschule bedeuten. 
\vfill
    \begin{flushright}
        Verabschiedet am 03.06.2018 in Heidelberg
    \end{flushright}
\end{document}
%%% Local Variables:
%%% mode: latex
%%% TeX-master: t
%%% End:
