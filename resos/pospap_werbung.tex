% !TEX TS-program = pdflatex
% !TEX encoding = UTF-8 Unicode
% !TEX ROOT = ../main.tex

\subsection*{Gegen kommerzielle Werbung in Lern- und Lehrräumen}
Die Zusammenkunft aller Physik Fachschaften (ZaPF)  spricht sich dafür aus, dass in Räumen der Lehre und des Lernens (z.B. Bibliotheken, Hörsäle, Übungsräume, Praktikumsräume) bei Lehr- und Lernbetrieb das Arbeiten ohne Beeinflussung durch Werbung stattfinden soll. Sinn der Lehrveranstaltungen und des Lernbetriebs ist es, dass Studierende unbeeinflusst von Interessen Dritter Fachinhalte erlernen und diskutieren, sowie Lehrende Lehrinhalte frei vermitteln können. Diese Arbeitsatmosphäre wird durch Werbung beeinträchtigt.
Kommerzielle Werbung\footnote{Werbung meint hier Maßnahmen zur Öffentlichkeitswirkung von kommerziellen, außeruniversitären Einrichtungen.}  in diesen Räumen, insbesondere Hörsaal- und Raumbranding\footnote{Hörsaal- und Raumbranding meint hier den Verkauf von Namensrechten von Hörsälen und anderen Lehr- und Lernräumen. In konkreten Fällen kann dies das Anbringen von Firmenlogos am und im betroffenen Raum und an der Rauminfrastruktur, sowie die Eintragung des Namens ins Raumverwaltungssystem der Hochschule bedeuten.} ist daher nicht hinnehmbar.