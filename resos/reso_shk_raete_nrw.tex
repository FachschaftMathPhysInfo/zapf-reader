\documentclass[DIV=calc]{scrartcl}
\usepackage[utf8]{inputenc}
\usepackage[T1]{fontenc}
\usepackage[ngerman]{babel}
\usepackage{graphicx}

\usepackage{ulem}
%\usepackage[dvipsnames]{xcolor}
\usepackage{paralist}
\usepackage{fixltx2e}
%\usepackage{ellipsis}
\usepackage[tracking=true]{microtype}

\usepackage{lmodern}                        % Ersatz fuer Computer Modern-Schriften
%\usepackage{hfoldsty}

%\usepackage{fourier}             % Schriftart
\usepackage[scaled=0.81]{helvet}     % Schriftart

\usepackage{url}
\usepackage{tocloft}             % Paket für Table of Contents

\usepackage{xcolor}
\definecolor{urlred}{HTML}{660000}

\usepackage{hyperref}
\hypersetup{
    colorlinks=true,    
    linkcolor=black,    % Farbe der internen Links (u.a. Table of Contents)
    urlcolor=black,    % Farbe der url-links
    citecolor=black} % Farbe der Literaturverzeichnis-Links

\usepackage{mdwlist}     % Änderung der Zeilenabstände bei itemize und enumerate
\usepackage{draftwatermark} % Wasserzeichen ``Entwurf'' 
\SetWatermarkText{}%Entwurf}

\usepackage{blindtext}
\parindent 0pt                 % Absatzeinrücken verhindern
\parskip 12pt                 % Absätze durch Lücke trennen

%\usepackage{titlesec}    % Abstand nach Überschriften neu definieren
%\titlespacing{\subsection}{0ex}{3ex}{-1ex}
%\titlespacing{\subsubsection}{0ex}{3ex}{-1ex}        

% \pagestyle{empty}
\setlength{\textheight}{23cm}
\usepackage{fancyhdr}
\pagestyle{fancy}
\cfoot{}
\lfoot{Zusammenkunft aller Physik-Fachschaften}
\rfoot{www.zapfev.de\\stapf@zapf.in}
\renewcommand{\headrulewidth}{0pt}
\renewcommand{\footrulewidth}{0.1pt}
\newcommand{\gen}{*innen}

\begin{document}
    \hspace{0.87\textwidth}
    \begin{minipage}{120pt}
        \vspace{-1.8cm}
        \includegraphics[width=80pt]{logo.pdf}
        \centering
        \small Zusammenkunft aller Physik-Fachschaften
    \end{minipage}
    \begin{center}
        \huge{Resolution der Zusammenkunft aller Physik-Fachschaften}\vspace{.25\baselineskip}\\
        \normalsize
    \end{center}
    \vspace{1cm} 
    \section*{Zur Position der SHK-Räte im neuen Hochschulgesetz NRW}
Die ZaPF (Zusammenkunft aller Physik-Fachschaften) verurteilt die im neuen Landeshochschulgesetz
geäußerte Ansicht, dass die SHK-Räte einen Fremdkörper in einem System der Interessenvertretung darstellen, da diese im Landespersonalvertretungsgesetz explizit ausgenommen werden. Stattdessen fordert die ZaPF  die Mitglieder des Landtages
NRW dazu auf, sich für einen Fortbestand und einen Ausbau der Vertretung derer einzusetzen, die an den Hochschulen ihres
Landes als Studentische Hilfskräfte tätig sind und somit in vielfältiger Weise erst den qualitativ hochwertigen 
Universitätsbetrieb ermöglichen. Aus diesem Grund wendet sich die ZaPF auch an alle Physik-Fachschaften und bittet alle 
Studierende der Physik um die Unterstützung der Petition gegen diese Änderung des Hochschulgesetzes in NRW.
\vfill
    \begin{flushright}
        Verabschiedet am 03.06.2018 in Heidelberg
    \end{flushright}
\end{document}
%%% Local Variables:
%%% mode: latex
%%% TeX-master: t
%%% End:


