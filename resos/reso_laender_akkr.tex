% !TEX TS-program = pdflatex
% !TEX encoding = UTF-8 Unicode
% !TEX ROOT = main.tex

\subsection*{Zu länderspezifischen Rechtsverordnungen als Spezifizierung der MRVO}
Die Landtage veröffentlichen im Rahmen der Überarbeitung des deutschen Akkreditierungssystems, gemäß den Artikeln des Studienakkreditierungsstaatsvertrags, Rechtsverordnungen zur Akkreditierung.\\
Diese müssen in Kernpunkten übereinstimmen, um eine "`bundesweite Gleichwertigkeit von Studien- und Prüfungsleistungen sowie Studienabschlüssen und die Möglichkeit des Hochschulwechsels"' (Staatsvertrag §1 (2)) zu gewährleisten. Sie dürfen allerdings nach § 4 (6) des Staatsvertrags auch weiterführende Verordnungen hinsichtlich der Qualitätsüberprüfung erlassen.

Die ZaPF fordert, dass in den Länderspezifischen Rechtsverordnungen gemäß § 4 (3), einer entsprechend überarbeiteten Musterrechtsverordnung (MRVO) vorgreifend, die folgenden Punkte als stärkere Richtlinien festgeschrieben werden:

\begin{itemize}
\item Akkreditierungsfristen (MRVO § 26 (1))
	\begin{itemize}
    \item Eine Akkreditierungsfrist von 8 Jahren (MRVO § 26 (1)) für eine Erstakkreditierung ist zu lang. Für neueingerichtete Studiengänge fordert die ZaPF eine erstmalige Reakkreditierung ein Jahr nach Ablauf der Regelstudienzeit, spätestens nach 5 Jahren.
    \end{itemize}
\item Zusammenstellung von Gutachtergruppen (MRVO §25)
	\begin{itemize}
    \item Alle Gutachter*innen sollen im Bereich Akkreditierung geschult sein -- entweder durch ihre Erfahrung oder durch entsprechende Fortbildungsmaßnahmen (MRVO §25 (3)).
    \item Bei Akkreditierungen von Lehramtsstudiengängen (MRVO § 25 (1)) darf die Vertretung der Berufspraxis in der Gutachtergruppe nicht durch einen Vertreter*in der obersten Landesbehörde ersetzt werden, sondern soll um diese*n ergänzt werden.
    \end{itemize}
\end{itemize}
