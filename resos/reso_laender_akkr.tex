\documentclass[DIV=calc]{scrartcl}
\usepackage[utf8]{inputenc}
\usepackage[T1]{fontenc}
\usepackage[ngerman]{babel}
\usepackage{graphicx}

\usepackage{ulem}
%\usepackage[dvipsnames]{xcolor}
\usepackage{paralist}
\usepackage{fixltx2e}
%\usepackage{ellipsis}
\usepackage[tracking=true]{microtype}

\usepackage{lmodern}                        % Ersatz fuer Computer Modern-Schriften
%\usepackage{hfoldsty}

%\usepackage{fourier}             % Schriftart
\usepackage[scaled=0.81]{helvet}     % Schriftart

\usepackage{url}
\usepackage{tocloft}             % Paket für Table of Contents

\usepackage{xcolor}

\definecolor{urlred}{HTML}{660000}

\usepackage{hyperref}
\hypersetup{
    colorlinks=true,    
    linkcolor=black,    % Farbe der internen Links (u.a. Table of Contents)
    urlcolor=black,    % Farbe der url-links
    citecolor=black} % Farbe der Literaturverzeichnis-Links

\usepackage{mdwlist}     % Änderung der Zeilenabstände bei itemize und enumerate
\usepackage{draftwatermark} % Wasserzeichen ``Entwurf'' 
\SetWatermarkText{}%Entwurf}

\usepackage{blindtext}
\parindent 0pt                 % Absatzeinrücken verhindern
\parskip 12pt                 % Absätze durch Lücke trennen

%\usepackage{titlesec}    % Abstand nach Überschriften neu definieren
%\titlespacing{\subsection}{0ex}{3ex}{-1ex}
%\titlespacing{\subsubsection}{0ex}{3ex}{-1ex}        

% \pagestyle{empty}
\setlength{\textheight}{23cm}
\usepackage{fancyhdr}
\pagestyle{fancy}
\cfoot{}
\lfoot{Zusammenkunft aller Physik-Fachschaften}
\rfoot{www.zapfev.de\\stapf@zapf.in}
\renewcommand{\headrulewidth}{0pt}
\renewcommand{\footrulewidth}{0.1pt}
\newcommand{\gen}{*innen}

\begin{document}
    \hspace{0.87\textwidth}
    \begin{minipage}{120pt}
        \vspace{-2cm}
        \includegraphics[width=80pt]{logo.pdf}
        \centering
        \small Zusammenkunft aller Physik-Fachschaften
    \end{minipage}
    \begin{center}
        \huge{Resolution der Zusammenkunft aller Physik-Fachschaften}\vspace{.25\baselineskip}\\
        \normalsize
    \end{center}
    \vspace{0cm} 
\section*{Zu länderspezifischen Rechtsverordnungen als Spezifizierung der MRVO}
Die Landtage veröffentlichen im Rahmen der Überarbeitung des deutschen Akkreditierungssystems, gemäß den Artikeln des Studienakkreditierungsstaatsvertrags, Rechtsverordnungen zur Akkreditierung.\\ 
Diese müssen in Kernpunkten übereinstimmen, um eine "`bundesweite Gleichwertigkeit von Studien- und Prüfungsleistungen sowie Studienabschlüssen und die Möglichkeit des Hochschulwechsels"' (Staatsvertrag §1 (2)) zu gewährleisten. Sie dürfen allerdings nach § 4 (6) des Staatsvertrags auch weiterführende Verordnungen hinsichtlich der Qualitätsüberprüfung erlassen.

Die ZaPF fordert, dass in den Länderspezifischen Rechtsverordnungen gemäß § 4 (3), einer entsprechend überarbeiteten Musterrechtsverordnung (MRVO) vorgreifend, die folgenden Punkte als stärkere Richtlinien festgeschrieben werden:

\begin{itemize}
\item Akkreditierungsfristen (MRVO § 26 (1)) 
	\begin{itemize}
    \item Eine Akkreditierungsfrist von 8 Jahren (MRVO § 26 (1)) für eine Erstakkreditierung ist zu lang. Für neueingerichtete Studiengänge fordert die ZaPF eine erstmalige Reakkreditierung ein Jahr nach Ablauf der Regelstudienzeit, spätestens nach 5 Jahren.
    \end{itemize}
\item Zusammenstellung von Gutachtergruppen (MRVO §25)
	\begin{itemize}
    \item Alle Gutachter*innen sollen im Bereich Akkreditierung geschult sein -- entweder durch ihre Erfahrung oder durch entsprechende Fortbildungsmaßnahmen (MRVO §25 (3)).
    \item Bei Akkreditierungen von Lehramtsstudiengängen (MRVO § 25 (1)) darf die Vertretung der Berufspraxis in der Gutachtergruppe nicht durch einen Vertreter*in der obersten Landesbehörde ersetzt werden, sondern soll um diese*n ergänzt werden.
    \end{itemize}
\end{itemize}
\vfill
    \begin{flushright}
        Verabschiedet am 03.06.2018 in Heidelberg
    \end{flushright}
\end{document}
%%% Local Variables:
%%% mode: latex
%%% TeX-master: t
%%% End:



\grid
