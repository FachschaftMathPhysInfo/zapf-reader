% !TEX TS-program = pdflatex
% !TEX encoding = UTF-8 Unicode
% !TEX ROOT = main.tex

\subsection*{Für einen flexibleren Umgang mit Prüfungsan- und abmeldungen}
Die ZaPF fordert, dass bestehende Systeme zur Prüfungsan- und abmeldung
überarbeitet und flexibel gestaltet werden.

Prüfungsan- und abmeldungen werden von Hochschulen individuell
gehandhabt und
dienen oft einem logistischem Zweck. Dies geht teilweise soweit, dass selbst
innerhalb einer Hochschule oft deutliche Unterschiede zu vermerken sind.
Hier stehen die Fristen im Widerspruch zu Flexibilität und
Studierendenfreundlichkeit.
Diese Fristen werden oftmals mit Raumplanung und organisatorischen
Problemen begründet.
Das Beispiel des Fachbereichs Physik der Freien Universität Berlin, in der es
keine verpflichtende Prüfungsanmeldung gibt und eine Prüfungsteilnahme als
Anmeldung gilt, zeigt jedoch, dass solche Begründungen hinfällig
sind. Außerdem können durch solche Maßnahmen Prüfungsämter
entlastet werden, da weniger irreguläre Abmeldungen anfallen.

In unseren Augen gibt es keinen Grund, warum Studierende zum Teil
mehrere Monate
vor Prüfungstermin von einer Prüfungsanmeldung zurücktreten müssen und
wir sehen
in dieser Form der Handhabung unnötige Hürden für Studierende.
Eine Prüfungsanmeldung soll, falls sie denn explizit nötig ist,
revidierbar sein.
Diese Revision sollte so spät wie möglich vor der Prüfung durchführbar sein.

Die Prüfungsvorbereitungszeit zwischen einer frühen Anmeldung und
Prüfung selbst
kann in vielerlei Hinsicht unverschuldet behindert werden. Daher wird
durch eine
Prüfungsanmeldung etliche Wochen vor der Prüfung, ohne eine Möglichkeit sich
abzumelden, den Studierenden die Flexibilität genommen, sich selbstsicher
für Prüfungen anmelden zu können.

Gerade hinsichtlich limitierter Prüfungsversuche, die an den meisten
Hochschulen
bedauerlicherweise praktiziert werden, ist eine solche Regelung
- vor allem im Zusammenhang mit Zwangsanmeldungen für den nächstmöglichen
Termin - eine absolute Zumutung.
In diesem Zusammenhang verweisen wir auch auf unsere Stellungnahme zur
Zwangsexmatrikulation aus Siegen\footnote{\url{https://zapfev.de/resolutionen/wise17/Zwangsexmatrikulation/Zwangsexmatrikulation.pdf}} im Wintersemester 17/18, in der sich die ZaPF gegen
jede Art von Zwangsmaßnahmen ausspricht.

Nur ein flexibles Anmeldesystem kann dem Bild einer fortschrittlichen
Hochschule
entsprechen, daher sieht die ZaPF die absolute Notwendigkeit, bestehende
An- und
Abmeldesysteme anzupassen.
Reguläre Prüfungsan- und abmeldungen müssen deshalb kurzfristig möglich sein und insbesondere Zwangsanmeldungen ohne die Möglichkeit des Rücktritts sind
grundsätzlich abzulehnen.
