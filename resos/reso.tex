\documentclass[DIV=calc]{scrartcl}
\usepackage[utf8]{inputenc}
\usepackage[T1]{fontenc}
\usepackage[ngerman]{babel}
\usepackage{graphicx}

\usepackage{ulem}
\usepackage[dvipsnames]{xcolor}
\usepackage{paralist}
\usepackage{fixltx2e}
\usepackage{ellipsis}
\usepackage[tracking=true]{microtype}

\usepackage{lmodern}                        % Ersatz fuer Computer Modern-Schriften
\usepackage{hfoldsty}

\usepackage{fourier}             % Schriftart
\usepackage[scaled=0.81]{helvet}     % Schriftart

\usepackage{url}
\usepackage{tocloft}             % Paket für Table of Contents

\usepackage{xcolor}
\definecolor{urlred}{HTML}{660000}

\usepackage{hyperref}
\hypersetup{
    colorlinks=true,    
    linkcolor=black,    % Farbe der internen Links (u.a. Table of Contents)
    urlcolor=black,    % Farbe der url-links
    citecolor=black} % Farbe der Literaturverzeichnis-Links

\usepackage{mdwlist}     % Änderung der Zeilenabstände bei itemize und enumerate
%\usepackage{draftwatermark} % Wasserzeichen ``Entwurf'' 
%\SetWatermarkText{Entwurf}

\usepackage{blindtext}
\parindent 0pt                 % Absatzeinrücken verhindern
\parskip 12pt                 % Absätze durch Lücke trennen

%\usepackage{titlesec}    % Abstand nach Überschriften neu definieren
%\titlespacing{\subsection}{0ex}{3ex}{-1ex}
%\titlespacing{\subsubsection}{0ex}{3ex}{-1ex}        

% \pagestyle{empty}
\setlength{\textheight}{23cm}
\usepackage{fancyhdr}
\pagestyle{fancy}
\cfoot{}
\lfoot{Zusammenkunft aller Physik-Fachschaften}
\rfoot{www.zapfev.de\\stapf@zapf.in}
\renewcommand{\headrulewidth}{0pt}
\renewcommand{\footrulewidth}{0.1pt}
\newcommand{\gen}{*innen}

\begin{document}
    \hspace{0.87\textwidth}
    \begin{minipage}{120pt}
        \vspace{-1.8cm}
        \includegraphics[width=80pt]{logo.pdf}
        \centering
        \small Zusammenkunft aller Physik-Fachschaften
    \end{minipage}
    \begin{center}
        \huge{Resolution der Zusammenkunft aller Physik-Fachschaften}\vspace{.25\baselineskip}\\
        \normalsize
    \end{center}
    \vspace{1cm} 
   \textbf{\large{Zum Streik der studentischen Hilfskräfte in Berlin}}
    

Es liegt im Verantwortungsbereich der Hochschulen den	reibungslosen
Lehrbetrieb	sicherzustellen.\\
Dafür sind die studentischen Hilfskräfte an	Hochschulen	und
Universitäten	unverzichtbar, weshalb sie als vollwertige Beschäftigte	der	Hochschulen	angesehen
werden	müssen.	Um	den	Studierenden	eine	Tätigkeit	an	den	Hochschulen	und	Universitäten	parallel
zu	ihrem	Studium	zu	ermöglichen	ist	es	notwendig,	dass	eine	ausreichende	Bezahlung	erfolgt	und
diese	regelmäßig	an	steigende	Lebenshaltungskosten	angeglichen	wird.

Daher	fordert	die	ZaPF , die Verhandlungsführenden der Arbeitgeber auf,	endlich	ein	faires	Angebot	vorzulegen	und	somit	eine
baldige	Einigung	in	den	Tarifverhandlungen	zu	erreichen.	In	der	Zwischenzeit	müssen	die
Hochschulen	allen	Studierenden	einen	regulären	Studienfortschritt	ermöglichen.

Vor	diesem	Hintergrund	solidarisiert	sich	die	ZaPF	mit	den	Studentischen	Beschäftigten	in	Berlin
und	ihrem	aktuellen	Arbeitskampf.


    \begin{flushright}
        Verabschiedet am 03.06.2018 in Heidelberg
    \end{flushright}
    
 %  \newpage 
%     \textbf{Antragsteller:} Christian	(Marburg),	Jenny	(FU	Berlin),	Timo	(TU	Berlin)	aus	dem	AK	Studentische	Tarifverträge
    
%    \textbf{Adressaten:} Präsidien der Berliner Hochschulen, der regierende Bürgermeister, Die	Verhandlungsführenden:
%    \begin{itemize}
%    \item FU:	Matthias	Dannenberg	(vk@ fu-berlin.de)
%    \item HU:	Andreas	Kreßler	(kressler@ uv.hu-berlin.de)
%	\item ASH	und	für	die	(Fach-)Hochschulen:	Andreas	Flegel	(kanzler@ ash-berlin.eu)
%	\item TU:	Beate	Niemann-Wieland	(beate.niemann@ tu-berlin.de),	
%	\item Benjamin	Klingbeil(benjamin.klingbeil@ tu-berlin.de)
%	\item UdK:	Claudia	Grotti	(claudia.grotti@ intra.udk-berlin.de)
%	\item Kommunaler	Arbeitgeber-Verband	(KAV):	Claudia Pfeiffer (claudia.pfeiffer@ kavberlin.de)
%    \end{itemize}

\end{document}
%%% Local Variables:
%%% mode: latex
%%% TeX-master: t
%%% End:

