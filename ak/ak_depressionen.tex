% !TEX TS-program = pdflatex
% !TEX encoding = UTF-8 Universitätcode
% !TEX ROOT = main.tex

\section{AK Depressionen im Studium}

  \textbf{Protokoll vom:} 30.05.2018,
  Beginn: 16:30 Uhr,
  Ende: 18:30 Uhr \\
  \textbf{Redeleitung:} Tobias Löffler (Universität Düsseldorf) \\
  \textbf{Protokoll:} Anna (Universität Kiel) \\
  \textbf{anwesende Fachschaften:} Universität Freiburg, Universität Freiberg, Universität Dresden, Universität Essen, Universität Chemnitz, Universität Bonn, Universität Bochum, Universität Lübeck, Universität Erlangen, Universität Saarbrücken,
  HU Berlin, Universität Potsdam, Universität Konstaz, Universität Düsseldorf, Universität Dordmundt, Universität Magdeburg, Universität Tübingen, Universität Münster,
  KIT, Universität Köln, Universität Frankfurt, LMU München, TU München, Universität Siegen, Universität Göttingen,
  Universität Bielefeld, RWTH Aachen, FU Berlin, Universität Wien, Universität Würzburg, Universität Rostock,

  \subsection*{Informationen zum AK}
    \begin{itemize}
      \item \textbf{Ziel des AKs}: Leitfaden für Fachschaftler, Broschüre/Wandzeitung in der Fachschaft
      \item \textbf{Folge-AK}: nein
      \item \textbf{Zielgruppe}: alle ZaPFika
      \item \textbf{Vorbereitung}: Deutsche Depressionshilfe und Bundespsychotherapeutenkammer bieten Infos und Überblick
    \end{itemize}

  \subsection*{Einleitung}
    Depressionen sind ein immer größer werdendes Phänomen für StudentInnen und damit auch für Physik-Studierende.
    Es gilt also zum Einen für das Thema zu sensibilisieren, aber auch darum den Fachschaftsräten, die ja nun mal keine Psychologen sind,
    eine Hilfestellung an die Hand zu geben, um den Wunsch nach Hilfe in diesem Thema zu unterstützen.

  \subsection*{Protokoll}

    Die Protokollierenden sind keine Experte/Expertin in dem Thema. Daher wird um Nachsicht bei möglichen ungünstigen Formulierungen gebeten. Es wurde nach bestem Wissen und Gewissen die Diskussion mitprotokolliert. \\

    Tobi gibt eine kleine Einleitung zum Thema. In diesem AK soll es speziell um diesen Teilbereich gehen: \\

    Awareness schaffen innerhalb der Fachschaften: Es gibt Depressionen an der Universität und Leute mit Depressionen sollen sich nicht allein gelassen fühlen, keine Stigmatisierung.
    Ein Hilfe zur Selbsthilfe: Es gibt bereits eine Menge Angebote für Menschen mit psychischen Problemen, jede Universität (zumindest in Deutschland). Wie gehe ich als Außenstehender an die Thematik heran? Wie erkenne ich Depressionen, wo kann ich helfen, wo nicht?
    Tobi fragt in die Runde, warum die Leute hier sind. \\

    Depressionen sind oft Grund von Abbrüchen, man will helfen.
    Die Quote der psychisch kranken Menschen ist im Studium durchaus signifikant.
    Man will eine Handreichung erarbeiten, wie es weiter geht, wenn das Problem fest gestellt ist.
    Man fühlt sich unbeholfen, weil Betroffene sich oft zurückziehen und somit aus dem wichtigen sozialen Umfeld heraus fallen. (Nicht immer so. Achtung: Stigmatisierung!)
    Menschen mit Depressionen tun sich schwer, sich einzugestehen, dass sie Depressionen haben. Das kann man von außen nicht beurteilen.
    Unterscheidung wichtig, ob sie wirklich depressiv sind oder eine Phase haben, in der es ihnen anderweitig schlecht geht.
    Wenn Menschen mit Depression weiter eingeladen werden zu Partys ist das an sich cool, aber das Umfeld muss sich auch eingestehen, dass man nicht immer helfen kann.
    Offen darüber reden heißt nicht gezielt mit jemandem über sein/ihr Problem reden, sondern eine gewisse Normalität zu spiegeln, man kann mit Menschen darüber reden.
    USA: Dort ist niederschwellige Therapie sehr viel normaler als hier. Vielleicht kann man das hier adaptieren, von wegen.
    Große Problematik ist, wenn man zum Studium umzieht, da dies eine Veränderung von Umgebungen und Therapie zur Folge hat. Und man in Folge des Umzugs sich vielleicht einredet, dass man an diesem neuen Ort ein neuer Mensch ist, und sich vor der Realität verstecken. Dies ist ein schwieriger Prozess, und man kann auf frühere Therapiemethoden hinweisen.
    Man soll sich jemanden gegenüber immer offen zeigen, wenn die Person einem etwas anvertraut, man sollte vorsichtig sein Diagnosen zu stellen. Es gibt gerade im Universitäts-Umfeld Beratungsangebote, die besser geschult dafür sind, also besser geeignet.
    Die niederschwellige Hilfestellung, die man selbst anbietet (auch als Fachschaft) kann sehr hilfreich sein, das ist ein nicht zu unterschätzender Faktor.
    Was ist Hilfe eigentlich? Zuhören ist sicherlich gut. Ist es besser jemanden zu sagen, ‘Du hast vielleicht Depression’ oder eher kleinreden/mit Samthandschuhen anzufassen?
    Gut gemeinte Ablenkung ist nicht unbedingt hilfreich.
    Es ist wichtig zu beachten, dass es um den Menschen geht und nicht den Depressiven.
    Es ist einfacher sich jemandem zu öffnen, den man nicht so oft sieht, als jemandem, den man täglich sieht.
    Depression lässt sich nicht verallgemeinern und ist bei jeder Person anders.
    Aus Erfahrung: Man kann psychologische Beratung gut weiter empfehlen. Wenn jemand nicht weiß, was mit ihm los ist.
    Was sind die Ursachen dafür, dass so viele Menschen an Depression leiden? Haben wir zu viel Druck in der Gesellschaft/im Studium? Dieser Punkt wurde explizit aus dem AK rausgenommen, weil dies den Rahmen sprengt.
    Man soll bitte bedenken, dass es nicht einfach ist, sich über diese Thematik zu öffnen.
    Hilfe anbieten, zu einer Beratungsstelle zu gehen, oder gemeinsam die Telefonnummer rauszusuchen, ist ein guter Anfang.
    Der betroffenen Person wiederspiegeln, dass sie OK ist (Normalität) und ihr ein offenes Ohr anbieten.
    Es wird festgestellt, dass gerade in der Diskussion verschiedene Fäden verfolgt werden:
    \begin{enumerate}
      \item Wie kann man den Raum schaffen, über diese Thematik zu reden?
      \item Wie kann man helfen, wenn jemand wirklich mit dieser Problematik zu der Fachschaft kommt?
      \item Wie kann man ``privat'' als Person, die Hilfestellung gibt, damit umgehen?
    \end{enumerate}
    Die Person verändert sich nicht, dadurch dass man jetzt weiß, dass sie depressiv ist. Es ist durchaus in Ordnung zu sagen, dass man mit der Situation überfordert ist und offen drüber zu reden, wie man ihr helfen kann. Aber die Person ändert sich nicht.
    Depression ist etwas individuelles, es gibt sehr unterschiedliche Formen, besser mit der betroffenen Personen zu reden und individuelle Umgangsformen zu finden.
    Tobi sucht eine kleine Gruppe, um einen Awareness-Flyer für alle Fachschaften zu verfassen. Á la: im Übrigen gibt es Depressionen, bitte achtet darauf.
    Als Nächstes wird darüber geredet, wie man als Fachschaft den Raum schaffen kann, darüber zu reden. Beziehungsweise, wie man als Fachschaft die Awareness schaffen kann. \\

    Bei der Erstsemester-Veranstaltung könnte man direkt darauf aufmerksam machen, dass es Depressionen gibt und dort auch auf Anlaufstellen aufmerksam machen (und auch jemanden von dieser Beratungsstelle als Referenten einladen).
    Man kann einmal im Semester oder im Jahr eine Infoveranstaltung zum Thema machen, eventuell mit Beratungsmenschen der Universität.
    Es wird kritisch gesehen, ob eine einzelne Infoveranstaltung am Beginn des Studiums sinnvoll ist. Es ist wichtig, konstant auf die Anlaufstellen aufmerksam zu machen.
    Es ist wichtig, das Gefühl zu vermitteln, dass man mit dieser Thematik nicht alleine ist und auch nicht allein gelassen wird.
    Menschen mit Depression, die sich offen dazu äußern können/wollen, sind hilfreich für Menschen ohne Depression, um ein Gefühl dafür zu bekommen, was Depression bedeutet.
    Infos über Anlaufstellen im Erstiheft/Flyer festhalten $\Rightarrow$ Möglichkeiten kommunizieren.
    Es ist wichtig, aktiv zuzuhören und anwesend zu sein.
    Depression kann auch die Form lauter Aggressivität annehmen.
    Persönlichkeitveränderungen sind etwas, auf das man generell achten sollte.
    Als Person, die helfen möchte, kann es sehr frustrierend sein, wenn die angebotene Hilfe nicht angenommen wird. In diesem Fall, darf man weder sich noch der anderen Person die Schuld geben. In diesem Fall kann man auch selber die Beratungsangebote annehmen. Es hilft niemandem, wenn man als Zuhörer selber Probleme bekommt oder es einen selbst belastet.
    Es wird vorgeschlagen, auf den Toiletten Werbung für Anlaufstellen zu machen.
    Es wird die Sorge geäußert, dass eine so frühe Thematisierung die Studierenden nur verschreckt und nicht hilfreich ist.
    Aus eigener Erfahrung ist es durchaus wichtig, früh auf diese Thematik aufmerksam zu werden.
    Es ist wichtig, dieses Thema zu normalisieren und je früher man für diese Thematik sensibel ist, desto besser. Es ist wichtig, den gesellschaftlichen Druck wegzunehmen.
    Im ersten Semester verändert sich eine Menge im persönlichen Leben, und dies kann durchaus Auslöser für eine Depression sein.
    Erstiheft wird durchaus gelesen; Es gibt auch Campusführungen, dabei kann man auf den Ort aufmerksam machen (die Beratungsstelle), damit das ganze nicht schief läuft, einfach ohne Scherz und mit nüchterner Erklärung vortragen.
    Kommunizieren, dass man nicht allein ist. Das kann passieren. Depression ist eine Krankheit, für die man sich Hilfe holen kann. Es ist ein normales Thema.
    Als AnsprechpartnerIn ruhig mit der Thematik umgehen, keine Dramatik erzeugen. Kommunizieren, dass die Universität ein Ort ist, an dem man sich entwickelt, nicht nur fachlich, sondern auch in der eigenen Entwicklung.
    Frage: Ist Arbeit ein Defence-Mechanismus, um sich von der Thematik abzulenken, und gleichzeitig ein Teufelskreis? Und sich somit nicht mehr auf sein Innenleben konzentrieren kann?
    Die Lernambulanz ist Anlaufstelle, um mit Klausurenstress umzugehen.
    Schulung für Fachschaftsräte von der Universität oder auch von Krankenkassen angeboten.
    Man kann dann gleich auch darüber berichten, und dadurch Awareness schaffen.
    Man kann auch zu Diakonie, Caritas oder Seelsorge Einrichtungen (kirchlich und nicht kirchlich) $\rightarrow$ die können einem entweder direkt weiterhelfen oder sie wissen, wo man Menschen für eine Schulung findet.
    Bei wie vielen Universitäten kostet Beratung zur Selbsthilfe?: mindestens eine
    Ziel einer Einführungswoche ist es, sich besser untereinander kennen zu lernen, und da ist dies anzusprechen wichtig.
    Das Thema sollte nicht nur bei Erstiveranstaltungen aufkommen, sondern kontinuierlich über die Semester immer wieder kommuniziert werden
    Man sollte vielleicht auch mit dem Fachbereich gemeinsam an Möglichkeiten der Kommunikation reden.
    Bei diesem Modell steigt allerdings der Altersunterschied, und dies kann problematisch werden.
    Hilfe ist Hilfe und wer helfen mag soll helfen können.
    Auch kann es helfen, jemanden als Ansprechpartner zu haben, der in seinem Leben erfolgreich ist und trotzdem oder deswegen Empathie zeigen kann.
    Modell der Vertrauenspersonen, die vielleicht auch geschult sind, zeigt, dass man sich mit der Thematik auseinander gesetzt hat und, dass man sich sorgt. Dies kann die Atmosphäre erheblich verbessern.
    Da Depression sehr individuell sein kann, sollte auch die Hilfe sehr breit aufgestellt sein. Jeder so, wie er kann und will. Dass man helfen will, ist Signal genug.
    Auf unterschiedlichen Kanälen sollte das Thema kommuniziert werden.
    Menschen, die möglicherweise betroffen sind, kann man ganz gut helfen, indem man bei der Selbsthilfe hilft: Beim Gang zur Hilfsstelle unterstützen.
    Ja, macht Sinn der Person dabei zu helfen. Das innere Problem, dass einen nicht handlungsfähig macht, kann ausgehebelt werden, wenn man als Betroffener noch jemanden hat, der mitgeht.
    ja, aber bereitet euch darauf vor, dass eure Angebote nicht angenommen werden, das hat aber nichts mit euch selbst zu tun, sondern ist meist Teil des Problems.
    Insbesondere wenn mögliche betroffene Personen Einem Nahe stehen, läuft man dadurch sehr schnell Gefahr, dass es einen selbst sehr stark belastet.
    Das Modell des Vertrauensdozenten wird vorgestellt. Diesen auch möglichst bald vorstellen.
    Auch außerhalb der Universität den Weg zur Therapie darstellen. Vielleicht fällt es den betroffenen Personen leichter, den Weg der Hilfe außerhalb der Universität zu gehen.
    Oft haben die Anlaufstellen innerhalb der Universität nicht die Möglichkeiten selber weiter zu helfen.
    Man sollte die Möglichkeit im Kopf behalten, dass es durchaus schwierig ist, geht man in Therapie, diese länger wahrzunehmen. Es kann der Fall auftreten, dass man nach zwei, drei Terminen diese wieder abbricht. \\
    Was genau ist Ziel des AKs?
    Es wird der Vorschlag gemacht, für die nächste ZaPF einen AK zu planen mit Referenten ($\rightarrow$ Arbeitsauftrag an den StaPF) \\
    Zwei mögliche Folge-AKs: \\
    \begin{itemize}
      \item Ursachen von Depressionen erarbeiten (im Studium?)
      \item Wie geht man damit um?
    \end{itemize}

    Ist es sinnvoll auf Ursachen einzugehen? Es kann doch eine Menge verschiedene Ursachen von Depressionen geben.
    Wohl mehr der Sinn des AKs: Ursachen im Studium.
    Ermöglicht Bekämpfung von Ursachen.
    Es wird als zu optimistisch eingesehen, dass man die Ursachen bekämpfen kann. Es ist durchaus interessant sich mit dieser Thematik auseinander zu setzen, aber wir sind nur Physikstudierende und keine PsychologInnen.
    Ist Awareness Schaffen nicht eine Art von Ursachen bekämpfen? In dem Fall ist es durchaus sinnvoll, sich über die möglichen Ursachen von Depressionen im Studium zu informieren.
    Man kann den Folge-AK genau darauf ausrichten. Die anwesende Orga der nächsten ZaPF liegt diese Thematik am Herzen und ist interessiert, sich darum zu kümmern.
    Es wird auf die Auswirkungen von Fehlernährung hingewiesen, auch dies kann Stimmungsschwankungen hervor rufen.
    Wir können keinerlei Diagnosen erstellen.
    Ursachen sind eher so gemeint: Wir können bei unserer Hochschulpolitischen Arbeit stärker daran denken, Ursachen zu berücksichten (z.B. Stresslevel bei Studiengangsumgestaltung gering halten).
    Man kann sich auch mit den ganz praktischen Dingen des Studiums befassen:
    \begin{itemize}
      \item Attestpflicht
      \item Wie geht man mit Klinikaufhalten um? Von organisatorischer Seite
      \item Wie kann die Universität so gestaltet werden, dass sie ``barrierefrei'' für Menschen mit psychologischen Problemen ist?
      \item Wie kann man die ``Degradierung'' durch psychologische Probleme entschärfen?
      \item Anerkennung von psychologischen Problemen als Krankheit an Universitäten einfordern.
    \end{itemize}
    $\rightarrow$ Es gibt durchaus praktische Dinge, die wir einfordern können und sollten. \\

    Es gibt den Vorschlag, dass der Arbeitsauftrag möglichst weit gefasst wird, sodass Würzburg die Freiheit hat, in Zusammenarbeit mit lokalen Beratungseinrichtungen, einen WS zu erarbeiten (die kennen sich ja schließlich mit der Thematik aus). Dafür kann Würzburg aus diesem AK berichten.
    Gibt es eine Prävention vor Depressionen? $\rightarrow$ Frage an potentielle ReferentInnen in Würzburg: Nein \\
    Es hilft aber Sport zu machen (aus eigener Erfahrung, kein Anspruch auf Gültigkeit)
    Ein Rat zum Abschluss: Wenn in einem persönlichen Gespräch über Suizidgedanken spricht: Dies ist das einzige Thema, das man bitte nicht mit sich alleine rumträgt. Bitte ruft mindestens das Sorgentelefon an (es gibt noch viele andere Hotlines dazu, auch lokale).
