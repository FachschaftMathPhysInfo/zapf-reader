% !TEX TS-program = pdflatex
% !TEX encoding = UTF-8 Unicode
% !TEX ROOT = main.tex

\section{AK Hörsaal-Sponsoring}

	\textbf{Protokoll vom:} 02.06.2018,
	Beginn: 09:00 Uhr,
	Ende: 11:00 Uhr \\
	\textbf{Redeleitung:} Niklas Donocik (Uni Braunschweig) \\
	\textbf{Protokoll:} Niklas Westermann (Uni Konstanz) \\
	\textbf{anwesende Fachschaften:} Uni Braunschweig, Uni Konstanz, TU Freiberg, Uni Dresden, Uni Karlsruhe, Uni Potsdam, Uni Rostock, TU Kiel, TU Darmstadt, Uni Bielefeld, Uni Duisburg/Essen, Uni Frankfurt, Uni Bochum, Uni Würzburg, Uni Göttingen, Uni Essen, FU Berlin, Uni Kaiserslautern, Uni Münster

	\subsection*{Informationen zum AK}
		\begin{itemize}
			\item \textbf{Ziel des AKs}: Meinungsbild, ggf. in einem Positionspapier festgehalten
			\item \textbf{Folge-AK}:  nein
      \item \textbf{Materialien}: siehe Einleitung \hyperref{sec:Einleitung}
			\item \textbf{Zielgruppe}: Lehrämter oder Leute, die sich mit Leramt auskennen
			\item \textbf{Ablauf}: geleitete Diskussion
			\item \textbf{Voraussetzungen}: Körperliche und im Idealfall geistige Anwesenheit
		\end{itemize}

  \subsection*{Einleitung}
    \label{sec:Einleitung}
    \underline{Thema}: Quereinsteiger. \\
    Eine Karriere Webseite tituliert zur Zeit "Gerade Quereinsteiger haben zurzeit beste Chancen auf Beamtenstatus und lange Ferien.". \footnote{\label{foot:1}\url{http://www.karriere.de/karriere/lehramt-quereinsteiger-willkommen-8686/}} Gerade in den naturwissenschaftlichen Fächer ist der Anteil an Quereinsteigern besonders hoch \footnote{\url{https://www.welt.de/wirtschaft/karriere/bildung/article135595922/Wenn-der-Quereinsteiger-den-Lehrermangel-ausgleicht.html}}. In allen 16 Bundesländern werden zur Zeit Planstellen an Bewerber ohne zweites Staatsexamen vergeben [ebd.] (das dient erstmal als gröbste Definition des Begriffs "Quereinsteiger"). \\

    Wir wollen uns in diesem AK grundsätzlich über das Thema unterhalten, vielleicht auch eigene Definitionen der Begriffe Quer- und Seiteneinsteiger bringen, um die Thematik aus studentischer Sicht zu betrachten. Insbesondere wollen wir über die Auswirkungen auf die Entscheidung zum Lehramtsstudium reden ("wenn man auch ohne Lehramtsstudium ins Lehramt kommt, warum soll ich das dann machen") und wie wir uns als angehende Lehrer zu den Risiken und Chancen für das Berufs- und Gesellschaftsbild Schule äußern. \\

    Weiterführende Informationen erhält die Studie zu Quereinsteigern, die die DPG 2010 herausgegeben hat \hyperref{foot:1}.

  \subsection*{Protokoll}
    \subsubsection{Einstieg und Themenfindung}

      Thema ist Quereinsteiger (QE) im Lehramt, solch einen AK gab es noch nicht, deshalb können wir uns thematisch 'austoben'. Niklas wirft ein Papier an, dass die Zahlen der Quereinsteiger anwirft (Link siehe oben im Wiki). \\
      Definition eines X-Einsteigers: Ein \underline{Quereinsteiger} steigt ins Referendariat ein, ein \underline{Seiteneinsteiger} geht direkt in den Lehrdienst. \\
      Uns ist aufgefallen, dass es fast gleich viele Quereinsteiger, wie normale Absolventen des Lehramtes in der Physik gibt (die Zahlen sind seit 2003 stark gestiegen). \\
      \textbf{Problem}: Den Quereinsteigern fehlt die Didaktik, deshalb ist es konkretisierungswürdig. Wir sollten uns überlegen ob und wenn ja was wir machen wollen. \\
      Es wird vorgeschlagen aufzulisten, was man als QE für den Einstieg machen muss und die höchste und niedrigste Anforderung aufzuschreiben. Oder wir schrieben das auf, was wir für sinnvoll halten. \\
      Wir merken, dass relativ viele Nicht-LehrämtlerInnen im AK sitzen und fragen erst mal, warum diese überhaupt in den AK gekommen sind. Generell ist das Interesse am möglichen Quereinstieg hoch. Folgende Themen fallen:
      \begin{outline}
        \1 Informationen sammeln um sie weiter zu geben
        \1 Anforderungen definieren
        \1 Herausfinden, was die Anforderungen sind und diese kritisch beurteilen
        \1 Vorurteile gegenüber Quereinsteigern und deren Qualität
        \1 Wir wollen gute Physik-LehrerInnen
        \1 Wie ist der fachliche 'Empfang', d.h. was ist die Vorbereitung vor dem Referendariat
        \1 Was gibt es für Unterschiede in der Bezahlung/Verbeamtung? Lehrämtler haben mehr Zeit für ihre Ausbildung investiert, manche QE werden nicht verbeamtet
        \1 Wollen wir im Idealfall überhaupt weiterhin Quereinsteigerprogramme, wenn es genug Lehrer gibt?
          \2 Ja, alle Studenten sollten weiter die Freiheit haben sich später zu entscheiden
          \2 Ja, da damit auch 'gute' Lehrer in die Schule kommen.
          \2 Ja, aber man sollte die QE gut didaktisch vorbereiten.
      \end{outline}
      Es wird diskutiert, was wir als Thema der Diskussion des AKs nehmen. Da es keine Einigung gibt, ob man sich grundsätzlich über die Sinnhaftigkeit von Quereinstiegern unterhalten soll, folgendes:

    \subsubsection{Was fordern wir von Quereinsteigern an (didaktischer) Ausbildung?}
      \paragraph{Sammlung der Meinungen}
        \begin{itemize}
          \item QEs sollten nicht unbedingt die ganze Erziehungswissenschaft (EW) des Idealstudiengangs machen müssen, da das nicht unbedingt sinnvoll ist und die Personen, die schon fertig studiert haben sich nicht mit Bachelorstudierenden 'abgeben wollen'.
          \item Als QE sollte man auf alle Fälle die Motivation haben, auch die ganze EW und Fachdidaktik (FD) zu machen.
          \item Vereinzelte Kurse sind einfacher nachzuhören, als das Machen eines zweijährigen Studiums, gerade Geldtechnisch. Beides sollte an der Uni stattfinden.
          \item Was gäbe es für andere Möglichkeiten der Ausbildung? $\rightarrow$ Fortbildungen von anderen Lehrern
          \item Wenn wir die Ausbildung an der Uni nicht fordern, könnte man auch die Ausbildung aus der Uni auslagern. \\ \textit{Klarstellung}: Die Erziehungswissenschaften an den Unis beinhalten meist wesentlich mehr Praxis als Forschung.
          \item Es gibt die Meinung, dass ausgebildetete Fachwissenschaftler besser wissen, was sie wollen, als angehende Studenten
          \item Eine Uni bildet einen besseren Durchschnitt über die schulische Lehre ab, als einzelne Schulen und Lehrer
        \end{itemize}

      \paragraph{Meinungsbilder}
        \begin{outline}
          \1 Wollen wir die Ausbildung der QEs an der Uni und nicht an anderen Orten wie Schulen etc.? $\rightarrow$ 16
          \1 Wer ist der Meinung, dass die Ausbildung auch durch andere Lehrer vorgenommen werden kann? $\rightarrow$ 2
          \1 Wer will, dass die FD-Ausbildung Teil des Studienseminars für QE ist? $\rightarrow$ 3
        \end{outline}

    \subsubsection{Welche Themen sind die wichtigsten (in der Physik)?}
      \begin{itemize}
        \item Didaktik (Präkonzepte...)
        \item Bildungswissenschftliche Forschungsanteile können weg gelassen werden (eine Bestandsanalyse muss aber z.B. gekonnt werden)
        \item Didaktische Inhalte sollten gekonnt sein, aber die Anwendung ist nicht so relevant (da wir annehmen, dass man die Anwendung im Allgemeinen schon im Fachstudium lernt)
        \item Man sollte sich jedoch auch Gedanken über die Verallgemeinerung unserer Forderungen in Bezug auf alle Fächer machen, da nur der Physik-Bezug im Lehramt Forderungen auf zu geringer Ebene stellt, weil die Ausbildung zentralisiert ist.
        \item Angebote zur Pflicht-Weiterbildung sollten über einen längeren Zeitraum (als das Refrendariat) angeboten werden, um das Arbeitspensum zu entspannen.\\ \underline{Meinungsbild}: 20 Dafür, 1 Dagegen, Rest Enthaltung
        \item Wenn wir alle 120 Leistungspunkte fordern, brauchen wir keine extra Auslbildung für QE sondern einfach nur Anrechenbarkeit des Fachstudiums.
        \item Unterschied zwischen QE und normalen LehrämtlerInnen ist, dass der QE während der Ausbildung schon als Lehrer arbeitet.
        \item Die 'theoretische Praxis' (Schul-Praktika,...) ist weniger wichtig. Das Lernen von Versuchsdurchführungen/-evaluationen jedoch schon (Demonstrationspraktika)
      \end{itemize}

    \subsubsection{Ergebnis}
      Wir wollen in Würzburg einen Folge-AK, in dem folgende Punkte diskutiert werden sollen: \\
      \begin{itemize}
        \item Evtl. jemaden einladen, der seine Erfahrungen mitteilen soll. Aber eher: \textbf{Die Ausbildungsprogramme in den Bundesländern durchlesen und zusammentragen}
        \item Einen Ausbilder von Quereinsteigern einladen (besser), Podumsdiskussion machen (Seminarleiter, Rektoren, ...)
        \item Wir wollen uns die konkreten Programme anschauen.
      \end{itemize}
