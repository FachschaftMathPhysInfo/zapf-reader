% !TEX TS-program = pdflatex
% !TEX encoding = UTF-8 Unicode
% !TEX ROOT = main.tex

\section{AK Rote Fäden der Studienreform}

	\textbf{Protokoll vom:} 31.05.2018,
	Beginn: 09:10 Uhr,
	Ende: 11:00 Uhr \\
	\textbf{Redeleitung:} Michel Vielmetter (Uni Köln) \\
	\textbf{Protokoll:} Michel Vielmetter (Uni Köln) \\
	\textbf{anwesende Fachschaften:} Uni Augsburg, FU Berlin, Uni Wuppertal, HU Berlin, Uni Gießen, Uni Würzburg, TU Darmstadt, TU Dresden, TU Chemnitz, Uni Konstanz

	\subsection*{Informationen zum AK}
		\begin{itemize}
			\item \textbf{Ziel des AKs}: Austausch und Sortierung von Leuten, die an der Weiterentwicklung ihrer Studiengänge arbeiten (wollen)
			\item \textbf{Folge-AK}: ja (WiSe `17 Siegen)
			\item \textbf{Vorwissen}: alte Protokolle lesen (\url{https://zapf.wiki/WiSe17_AK_Rote_Fäden_der_Studienreform}, \url{https://zapf.wiki/SoSe17_AK_Rote_Faeden_der_Studienreform})
      % \item \textbf{Materialien}:
			\item \textbf{Zielgruppe}: alle
			\item \textbf{Ablauf}: Inputs zu einzelnen "Fäden", möglichst aus der Studienreformdebatte verschiedener Universitäten; Diskussion
			% \item \textbf{Voraussetzungen}:
		\end{itemize}

  \subsection{Einleitung}
    Im Rahmen der Bachelor-Master-Umstellung vor gut 10 Jahren haben sehr viele und weitreichende Änderungen an unseren Studiengängen auf einmal stattgefunden. Spätestens seit den Bildungsstreiks 2009 ist klar, dass die Ergebnisse nicht gerade ideal waren. Seitdem hat es an fast allen Universitäten zahlreiche größere oder kleinere Veränderungen an den Studiengängen gegeben. Wir meinen es ist Zeit, die mal Revue passieren zu lassen und ein bisschen prinzipieller zu reflektieren, zumal viele Überarbeitungen ohne philosophisch-theoretische Background-Diskussionen an Hand konkreter Ärgernisse und Schwierigkeiten des Alltages teils von der Hand in den Mund entwickelt wurden. \\ % dieser Satz ...

    Idee dieses Arbeitskreises ist es, dass einzelne Fachschaften in kurzen Inputs versuchen, rote Fäden/die Kernüberlegung hinter der bisherigen, aber auch angedachten Weiterentwicklung ihrer Studiengänge (ideologiekritisch) vor- und zur Diskussion zu stellen. Wenn Ihr dazu mit einem Input beitragen wollt, tragt Euch bitte in die Liste der roten Fäden ein. Wenn dabei zu viele "Fäden" heraus kommen sollten, werden wir zu Beginn kurz klären, welche Priorität haben und welche wir in einen Bier-AK und/oder Nachfolge-AK verschieben. \\

    Beim letzten Mal wurde überlegt, dass es sinnvoll ist, vor Ort Änderungen, Erfahrungen und auch die Debatten dahinter zu dokumentieren. Es wäre gut, mittelfristig eine uniübergreifende Sammlung davon anzulegen. Auch dies kann im AK diskutiert werden.

  \subsection*{Protokoll}
    Berichte von vergangenen AKs

    \paragraph{Themenvorschläge:}
      \begin{itemize}
        \item Entlastung der Erstsemester
        \item nichtphysikalisches Nebenfach
        \item geisteswissenschaftliches Lernen in Naturwissenschaften
      \end{itemize}

      \textbf{These:} Spiralcurricula sind eh schon umgesetzt, wenn man sie jetzt aktiv umsetzt könnte die Qualität der Lehre steigen. \\
      Beispiel: Theoretische Physik in Köln. Wurde oft umgewürfelt, je nachdem welcheR Dozierende grade die lauteste Stimme hatte (siehe Protokoll aus Siegen).

    \begin{outline}
      \1 FU Berlin: Hat lineare Algebra vorgezogen, womit Analysis weiter nach hinten gerutscht ist im Studienverlauf.
      \1 Augsburg: Viele schaffen das Studium aufgrund von Mathe nicht, die sollte deshalb nicht zu spät kommen
      \1 Würzbürg: Mathe für Physiker, nicht mit Mathematiker*innen. Funktioniert je nach dozierender Person. Aufbauende Mathe(DGL, Funktionentheorie) von Physikern.
      \1 Köln: Modelle für Matheeinbindung:
      \2 Integriert in Kooperation von Mathe und Physik; Viel Aufwand für Dozierende, da hier viele Absprachen getroffen werden müssen.
      \2 Physiker machen Mathe; Einige mathematische Zusammenhänge werden nicht gut dargestellt.
      \2 Mathe macht Mathe; Wenig Relevanz im physikalischen Sinne.
    \end{outline}

    Im Spiralcurriculum könnte eine Mischform gewählt werden: Erst PhysikerInnen machen Mathe: Für den einfachen und schnellen Einstieg, danach richtige Mathe.

    \begin{itemize}
      \item Darmstadt: Es gibt verschiedene Lerntypen. Diese sollten an verschiedenen Universitäten alle berücksichtigt werden.
      \item HU Berlin: Integrierter Kurs war zu anspruchsvoll, was ihn abgeschafft hat. Er wird aber nach wie vor unter der Hand umgesetzt.
      \item Köln: Ein Spiralcurriculum könnte auch verschieden Lernarten abbilden. Studierende könnten sich nach Belieben ihren Lernstil an den Anfang des Studiums legen.
      \item Gießen: Problem, wenn unterschiedliche Jahrgänge in einer Vorlesung sitzen? \\
            $\rightarrow$ Köln: Kommt auf die Uni an, wenn Puzzle-Studienverlauf, dann okay.
      \item Augsburg: Es kann gar nichts gewählt werden und baut aufeinander auf.
      \item Köln: Hat einen Studienplaner mit Abhängigkeiten, anstatt Musterstundenplan. Dieser ist in Säulen organisiert, die allerdings viele Abhängigkeiten haben. Wenn diese gekürzt oder im Studienverlauf nach hinten geschoben werden, wird die Übersicht gesteigert. Dies soll in der nächsten Reakkreditierung angesprochen werden.
            Pfeile können unterschiedlich dick sein. Also sind einige besser kürzbar als andere. Computerphysik ist relativ unabhängig. Mit Projektarbeiten könnte hier eine individuelle Abhängigkeit geschaffen werden. \\
            $\rightarrow$ FU Berlin: Man könnte Wolken an Abhängigkeiten schaffen, sodass man sich freier durch das Studium bewegt.

      \item Informationsbroschüren sollten auch diese Diversität des Studiums abbilden, sodass man als Studierender frei wählen kann.
      \item Köln: Hat viele verschieden Verlaufspläne erstellt, für verschiedene Lerntypen, oder auch wenn man bei einer Klausur durchgefallen ist.
      \item Chemnitz: Die Überforderung gehört auch mit zum Physikstudium.
      \item FU Berlin: Es ist falsch, wenn es eine zeitliche Überforderung gibt. Frustrationsgrenze muss durch das Studium gesteigert werden, aber man sollte dabei nicht im Stich gelassen werden.
      \item Gießen: Man muss lernen, dass man nicht alles sofort verstehen muss. Nicht alles beim ersten Mal verstehen, sollte noch nicht scheitern heißen.
      \item Köln: Die Leute wissen nicht, wo ein Scheitern (wie Klausurdurchfall) in Ordnung ist und wo sie mehr Zeit investieren müssen.
      \item FU Berlin: Die Kernkompetenzen werden nicht immer sichtbar. Hierauf sollte besser hingewiesen werden.
      \item Augsburg: Das gehört mit zur Studienleistung. Das sollten die Menschen auch selber lernen.
      \item Köln: In der Vorlesung wird noch mal mehr eingeordnet als beim Selber-Lernen. Dies sollte auch für diese Kernkompetenzen gelten.
      \item Chemnitz: Die schlechtesten Übungen sind die, wo nicht auf die Lernenden eingegangen wird.
      \item Köln: Man muss den Leuten nichts auf dem Silbertablet servieren. Aber die Lernbedingungen können so verändert werden, das man sich auf die wesentlichen Dinge konzentrieren kann und durch Raumplanung auch Interaktion und Kommunikation fördert.
      \item HU Berlin: Öffnung zum Sommer-Semester, Erstis (aus dem Sommer) hören erstmal drei Semester nur Experimentalphysik.
      \item Köln: nach 3-4 Wochen Austausch-Treffen mit den Erstis, manche Leute finden nicht direkt den Anschluss und informieren sich nicht selbst. Hier könnte ein Austauschseminar helfen, mit Anekdoten und allem drum und dran.
      \item Köln: Die Mathe bietet 6 verschiedene Pläne an, die gleichwertig sind. Sie haben auch in den ersten Semestern nicht viel Ähnlichkeit, damit die Studiereneden sich damit beschäftigen müssen, was sie wollen.
      \item Würzburg: Optionale, mehrsemestrige Module sind nicht wirklich machbar, was eine Freiheit einschränkt. Grade ein Praktikum kann man nicht zeitlich frei gestalten.
      \item Gießen: Die Voraussetzungen am Anfang des Studiums sind gar nicht so krass. Hier wäre eine freie Wahl schon recht leicht möglich.
      \item Augsburg: Die meisten Menschen studieren das gleiche. Warum muss man hier so viel schieben können sollen?
      \item Darmstadt: Es sollten schon die Grundlagen am Anfang gelernt werden.
      \item TU Dresden: (an Augsburg) Man sollte auf die Individualtät der Menschen eingehen. Nicht alle.
      \item Köln: Wenn man am Ende des Studiums Umfragen über den tatsächlichen Studienverlauf macht, hat dies fast nie etwas mit dem eigentlichen Verlaufsplan zu tun.
            Die Grundlagen unterscheiden sich von Uni zu Uni, was die Flexibilität zeigt.
      \item Würzburg: Wenn Regel-Studienverläufe nicht eingehalten werden, woran liegt dies? Bewusste Entscheidung oder durch Klausuren gefallen? \\ $rightarrow$ ja
      \item FU Berlin: Man sollte, egal welcher Grund, die Regelstudienverlaufsabweichler*innen an ihrem aktuellen Studienpunkt abholen können.
      \item Köln: Die Meisten versuchen den Studienverlaufsplan einzuhalten, und werden aus der Bahn gekegelt, worduch sie auf den Geschmack am Schieben kommen.
      \item Dresden: Bleiben Lerngruppen zusammen, obwohl so viel Flexibilität besteht? $\rightarrow$ möglich
      \item Würzburch: Wahlpflicht soll neu designed werden.
      \item FU Berlin: 30 Punkte Wahlbereich, wo es kaum Einschränkungen gibt. Das hat bisher sehr gut geklappt, da die Studierenden wissen, was sie wollen.
      \item Würzburg: Bayern mag nicht; Einschränkungen für alle
      \item Köln: Die Wahl soll extra einen Blick über den Tellerrand gewähren. Die Studierenden sollen nicht zu Fachidioten erzogen werden.
      \item Wuppertal: Vor allem Einschränkungen durch Prüfungsordnung, da man perfekt die vorgegebene Leistungspunkt(LP)-Zahl treffen muss. Auch wenn die Vorlesung mehr LP hat als angefordert, ist sie illegal.
      \item Würzburg: Freiheit befreit. Schön.
    \end{itemize}

    Der AK wird auf der nächsten ZaPF fortgesetzt.
