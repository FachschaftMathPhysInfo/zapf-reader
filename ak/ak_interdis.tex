% !TEX TS-program = pdflatex
% !TEX encoding = UTF-8 Unicode
% !TEX ROOT = main.tex

\section{AK Förderung der Interdisziplinarität und Modulgrößen}

  \textbf{Protokoll vom:} 02.06.2018,
  Beginn: 15:00 Uhr,
  Ende: 17:00 Uhr \\
  \textbf{Redeleitung:} Christian (Marburg) \\
  \textbf{Protokoll:} Christian (Marburg) \\
  \textbf{anwesende Fachschaften:} Augsburg, Bonn, Chemnitz, Cottbus, Dresden, Freiberg, Graz, Konstanz, Mainz, Marburg, Münster, Rostock, Saarland, Tübingen, Uni Wien, Würzburg

  \subsection*{Informationen zum AK}
    \begin{itemize}
      \item \textbf{Ziel des AKs}: Austausch der Umgangsformen und Entwickeln von Ideen
      \item \textbf{Folge-AK}: nein
      \item \textbf{Zielgruppe}: alle
      \item \textbf{Ablauf}: Austausch der verschiedenen Situationen und anschließend Diskussion einiger Ideen
    \end{itemize}

  \subsection*{Einleitung}
    Aktuelle Situation in Marburg:
    \begin{outline}
      \1 Das Unipräsidium hat beschlossen, dass alle Module nur noch 6 ECTS Punkte haben sollen
      \1 Gründe dafür werden genannt:
        \2 Bessere Modulaisierbarkeit (bessere Einteilung des Workloads)
        \2 Damit verbesserte Situation für Studierende im Bezug auf Lohnarbeit (Sozialerhebung).
        \2 Einheitliche Modulgrößen fördern Austausch von Modulen und die Interdisziplinärität.
      \1 Daraus ergeben sich folgende Fragen:
        \2 Sind einheitliche Modulgrößen sinnvoll?
        \2 Wie gut klappt der Import von Modulen aus anderen Fachbereichen (als Nebenfach, zusätzliches Modul)?
        \2 Wie stehen wir zu zusätzlichen Modulen über 180/240 CP hinaus, wie sollen die anerkannt werden (im Zeugnis auftauchen)?
        \2 Welche anderen Ideen haben wir, um die sinnvollen Ziele zu erreichen?
    \end{outline}

  \subsection*{Protokoll}
    \paragraph{Einführung}
      Erklärung, wie es in Marburg abläuft; Einführung von “Marvin” als neuem CMS System. \\
      Fehlschlag aufgrund von organisatorischen und systembedingten Fehlern (nicht importierte Module können nicht belegt werden). \\
      Alle Module sollen auf eine gleiche Anzahl an ECTS Punkten genormt werden (6 ECTS).
      Aber im Zuge dessen soll der Workload für einzelne Veranstaltungen angepasst werden. \\

      \textit{Ziel:} Austausch mit anderen Fachschaften, wie deren Haltung zu diesem Thema ist, um selbe ein breiteres Meinungsbild zu erhalten.

    \paragraph{Diskussion}
      \begin{outline}
        \1 Einschränkung nach 240 ECTS Punkten $\rightarrow$ keine Möglichkeit mehr mehr Vorlesungen etc. anrechnen zu lassen.
        \1 Motivation für Normierung der Module: Workload wird auf mehr Module verteilt und so das Zeitmanagment des Studiums während anderer Aktivitäten erleichtert. Die Organisation wird einfacher.
        \1 Meinung der Fachschaft Marburg dazu: Prinzipiell der Gedanke dahinter (bessere Modularisierbarkeit) wird als gut empfunden, nur die Umsetzung ist fragwürdig oder gar nicht möglich.
        \1 Wichtig: Diese Prüfungsordnung ist bereits beschlossen, das heißt Ziel des AKs ist nicht, wie man dagegen vorgehen kann, sondern wie man damit umgehen kann.
        \1 Wie ist die Wahlfreiheit des Bachelors bei anderen Universitäten? Austausch über verschiedene Bachelormodelle.
        \1 Zielsetzung: Die gegeben Tatsachen besser nutzen, um das Studium so frei wie möglich zu gestalten.
      \end{outline}

  \subsection*{Zusammenfassung}
    \begin{outline}
      \1 Einheitlicher Wahlpflichtbereich (dann ist Einbindung leichter) \\
        $\rightarrow$ Nicht alle stimmen zu.
      \1 Wenn Erwännungen im Zeugnis nicht möglich ist, eher Scheine für zusätzliche Veranstaltungen geben.
      \1 Andere Möglichkeiten, andere Studienfächer anrechnen zu lassen (Studium Generale, …)
        \2 Möglichst freie Wahl in diesem Bereich, hier sollte alles anrechenbar sein.
        \2 Autonome Tutorien
    \end{outline}
