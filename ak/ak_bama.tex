% !TEX TS-program = pdflatex
% !TEX encoding = UTF-8 Unicode
% !TEX ROOT = main.tex

\section{AK BaMa}

	\textbf{Protokoll vom:} 31.05.2018,
	Beginn:14:10 Uhr,
	Ende: 15:30 Uhr \\
	\textbf{Redeleitung:} Sonja (Uni Bonn / KommGrem) \\
	\textbf{Protokoll:} Merten (Uni Göttingen / jDPG) \\
	\textbf{anwesende Fachschaften:} Uni Würzburg, Uni Tübingen, Uni Wuppertal, Uni Düsseldorf, Uni Augsburg, Uni Potsdam, TU Darmstadt, TU Wien, Uni Gießen, Uni Bochum, Karlsruhe (KIT), Uni Dresden, Uni Cottbus, Uni Rostock, TU Graz, Uni Oldenburg, Uni Bielefeld

	\subsection*{Informationen zum AK}
		\begin{itemize}
			\item \textbf{Ziel des AKs}: Ausarbeitung der nächsten ToDos
			\item \textbf{Folge-AK}: ja
      \item \textbf{Materialien}: allgemein: \url{https://zapf.wiki/Bachelor-Master-Umfrage}, neue Umfrage: \url{https://zapfev.de/resolutionen/sose17/PosPapier_BaMa_Umfrage/PosPapier_BaMa_Umfrage.pdf}
			\item \textbf{Zielgruppe}: alle ZaPFika, die Wert auf eine faire Beurteilung der Studiengänge unabhängig vom CHE legen
			\item \textbf{Ablauf}: aktueller Stand, Diskussion der kommenden Umfrage
			\item \textbf{Voraussetzungen}: keine
		\end{itemize}

  \subsection*{Einleitung}
    Geschichtlicher Abriss: Sonja berichtet, davon was bisher geschah \url{https://zapf.wiki/Bachelor-Master-Umfrage}. \\
    \begin{itemize}
      \item Umfrage zuerst durchgeführt (durch ZaPF und jDPG), als die meisten Unis Bachelor/Master eingeführt haben, um zu “erforschen”, wie das ganze implementiert wurde
      \item Nach 4 Jahren wurde sie wiederholt, um die Entwicklung anzuschauen. Neuer Schwerpunkt: Studieneinstieg
      \item Umfrageergebnisse wurden auch für ZaPF-AKs genutzt
      \item Bislang keine große Veröffentlichung der Ergebnisse, da viel zu umfangreich
      \item Fachschaften können die Daten von ihrer eigenen Hochschule erfragen
      \item Es wurde PosPapier verabschiedet, dass wir die Umfrage auch in Zukunft alle vier Jahre wiederholen wollen
      \item Es soll einerseits Kernfragen geben, die immer wieder gefragt werden sollen, um langfristige Entwicklungen sehen zu können, andererseits Spezialfragen, die spezielle einzelne Themen thematisieren.
      \item In den lezten Wochen wurde relativ viel Zeit investiert, um auf Grundlage der bisherigen Fragebögen, neue Fragebögen zusammenzustellen
    \end{itemize}

  \subsection*{Protokoll}
      \subsubsection*{Fragen/Anmerkungen zum aktuellen Stand}
        \begin{outline}
          \1 Können die Ergebnisse auch für einzelne Hochschulen abgefragt werden?
            \2 Ja, insbesondere können auch Durchschnittsergebnisse einer Hochschule mit deutschlandweiten Durchschnittsergebnissen verglichen werden
            \2 Noch viel Arbeit zu tun, die Daten vernünftig zu speichern
          \1 Was hat es mit dem Block zu Ethik-Fragen auf sich?
            \2 Ethik-AKs haben sich über Jahre hinweg in diverse Richtungen entwickelt, allerdings war immer das Problem, dass man nicht wusste, was eigentlich von Seiten der meisten Studierenden gewollt ist und wie die Situation an einzelnen Unis aussieht.
            \2 Deshalb versuchen wir mit dieser Umfrage, Daten zu dem Thema zu erheben
          \1 Sichern (z.B. via Captcha?)
            \2 IP-Adressen speichern, Codes generieren ist impraktikabel
            \2 Captcha klingt nach einer sinnvollen Lösung
          \1 Was tun bei sehr kleinen Studiengängen?
            \2 Viele spezialisierte Masterstudiengänge haben nur ~15-20 Studierende
            \2 Bei zu wenig Rückläufern ist die Umfrage nicht mehr wirklich anonym
            \2 Umfrage soll nicht fachschaftseigene, studienspezifische Evaluationen ersetzen
          \1 Sind alle Fragen verpflichtend?
            \2 Nein, nur ganz grundlegende Fragen (Hochschule, Studiengang) sind Pflicht
        \end{outline}

      \subsubsection*{Heute zu tun}
        \begin{outline}
          \1 Zeitplan für die Umfrage
          \1 Erklärtext
          \1 Best Practices/Handreichung für Fachschaften
            \2Wie kann man in einer online-Umfrage den Rücklauf erhöhen?
        \end{outline}

        Die Testumfragen sollen noch während der ZaPF ausgefült werden:
        \begin{outline}
          \1 Studierendenbogen: \url{https://www.fsr.physik.uni-goettingen.de/ls/166771?lang=de-informal}
          \1 Hochschulbogen: \url{https://www.fsr.physik.uni-goettingen.de/ls/719595}
        \end{outline}
        Die einzelnen Fragen werden in diesem AK nicht besprochen, dafür ist ein Mumble-Treffen vorgesehen.

      \subsubsection*{Zeitplan}
        \begin{outline}
          \1 Links sollten mindestens drei Wochen offen bleiben
          \1 Ist realistisch, um den Link noch sinnvoll zu verteilen
          \1 Semester ist in NRW in der dritten Juli-Woche zu Ende, in Österreich bereits am 30.6.
          \1 Vorlesungszeit ist wahrscheinlich besser als vorlesungsfreie Zeit, da Studenten besser erreichbar
          \1 Prüfungszeiten können auch sinvoll sein, da man dort gerne fachfremde Dinge macht
          \1 Langfristig ist erstes Drittel des Semesters besser, jedoch in diesem Fall nicht praktikabel, noch bis zum Wise zu warten
          \1 Wann kann der Fragebogen bereitgestellt werden?
            \2 Theoretisch sollte die Woche vom 11.-17.6. machbar sein
          \1 Start-Termine
            \2 spätestens ab 18.6., idealerweise in der Woche davor
          \1 Nachbefragung entweder im Wintersemester oder kurz danach im Anschluss bei FSen mit geringem Rücklauf
        \end{outline}

      \subsubsection*{Kommunikation}
        \begin{outline}
          \1 Abschluss zum Ausfüllen der Umfrage als +2 Wochen kommunizieren, dann nochmals um 1 (2) Wochen verlängern.
          \1 Erinnerung mit aktuellem Rücklauf an die Fachschaften
          \1 E-Mail-Verteiler (ggf. Dekanate fragen)
          \1 Fachschaftsbroadcast
          \1 Facebook
          \1 Infos auf Übungszetteln
          \1 Start-Bildschirme in CIP-Pools
          \1 QR-Code/Link in der Vorlesung
            \2 Eher nicht in der Vorlesung selbst ausfüllen
            \2 In der Vorlesung Kurzzusammenfassung, was das eigentlich soll
          \1 Info-Bildschirme, schwarze Bretter etc.
        \end{outline}

      \subsubsection*{Ideen für Handreichung an Fachschaften}
        \begin{outline}
          \1 QR-Code (in Toilettenkabinen, “Klopapier”)
          \1 kurzer Link
          \1 Ausfüllbarkeit auf dem Handy
          \1 Wie Zugriff auf die Studierenden / E-Mails der Studierenden?
            \2 Facebook-Gruppen (Twitter, …)
            \2 Fachschafts-Broadcast
            \2 Anfragen über Dekanat / Verwaltung
            \2 Messenger-Gruppen (Telegram, WhatsApp, Signal, Threema, Rocket Chat.. )
        \end{outline}

      \subsubsection*{Was muss in die Handreichung?}
        \begin{outline}
          \1 Termin
          \1 Was soll das ganze eigentlich?
          \1 Geschichtlicher Abriss (s.o.)
          \1 FAQ für Studis und Fachschaften
        \end{outline}

      \subsubsection*{Still to do}
        \begin{outline}
          \1 Fragenübersetzung auf Englisch
          \1 Flyer gestalten
          \1 Datenauswertung ist noch ein großes Thema
            \2 Irgendwie den Teilnehmenden eine Möglichkeit bereitstellen, die Ergebnisse einzusehen
          \1 Es haben drei neue Personen Interesse, weiter zu arbeiten
        \end{outline}

  \subsection*{Zusammenfassung}
    \begin{outline}
      \1 Die BaMa Umfrage soll noch im Juni 2018 starten.
      \1 Es werden Informationen und Handreichungen für die Fachschaften vorbereitet, Ideen dazu wurden gesammelt.
      \1 Die Unterstützung aller Fachschaften wird für die Umfrage benötigt, in Form von Verteilung des Online-Link und Bewerbung der Umfrage.
      \1 Die Qualität der Fragebögen wird besser, wenn jetzt noch viele ZaPFika an der Test-Umfrage teilnehmen! (Die Ergebnisse zählen offensichtlich nicht in die eigentliche Umfrage)
      \1 Es haben sich neue Interessierte gefunden, es besteht weiterhin Bedarf nach mehr jungen LEUTE für HUMBUG zum treiben.
    \end{outline}
