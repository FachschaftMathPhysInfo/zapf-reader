% !TEX TS-program = pdflatex
% !TEX encoding = UTF-8 Unicode
% !TEX ROOT = ../main.tex

\section{AK Organizing an international welcome}

  \textbf{Protokoll vom:} 31.05.2018, %???
  Beginn: 16:30 Uhr, %???
  Ende: 18:30 Uhr \\ %???
  \textbf{Redeleitung:} Siddhu Chelluri (Uni Siegen) \\
  \textbf{Protokoll:} Lina Vandré (Uni Innsbruck) \\
  \textbf{anwesende Fachschaften:} Uni Göttingen, Uni Siegen, Uni Insbruck, Uni Gießen, Uni Würzburg, Uni Wien, Uni Dresden, LMU München, TU Ilmenau, TU Darmstadt, Uni Oldenburg, Uni Düsseldorf, FH Lübeck, HU Berlin, Uni Saarland, Uni Marburg, Uni zu Köln, Uni Münster, Alumni

  \subsection*{Informationen zum AK}
    \begin{itemize}
    	\item \textbf{Ziel des AKs}: Handout zur Durchführung kommender Veranstaltungen
    	\item \textbf{Folge-AK}: nein
      \item \textbf{Materialien}: Materialien von bereits existierenden Veranstaltungen
    	\item \textbf{Zielgruppe}: alle ZaPFika, die Erfahrung oder Interesse an der Organisation solcher Veranstaltungen haben.
    	\item \textbf{Ablauf}: Erfahrungsaustausch, anschließend Erstellen von Guidelines für solche Events
    	\item \textbf{Voraussetzungen}: keine
    \end{itemize}

  \subsection*{Einleitung}
    When an international (non european) student arrives in a german spoken country there are a lot of
    formalities and procedures to be done by them. While most of the student representatives (FSR, FSI, StV)
    are (german) native speakers, they don't have experiences with this things. This makes it difficult to
    organise an international welcome (such as Erstsemestereinführung, Orientierungswoche, Erstsemestrigentutorium...). \\

    In the AK we want to discuss what is nessesary to tell the new students in such a welcome event (content, not form).
    One aim is to write a guideline/handout which shall contain all the important information regarding the first steps
    at the university and in the country. It contains the general information such as bank account, visa extension etc.
    Furthermore there is also a section in the handout where we discuss and list the things and procedures that
    differ from university to university (such as library card, how to find courses etc) so that each FSR/FSI/StV...
    can make their own handouts for their unique university regulations. \\

    We will also have some discussion time where we can ask questions, share experiences and further ideas
    to improve it. We prepared it by taking the past experiences of the internaitonal students
    along with our own experiences and also suggestions from various people and resources. \\

    To get a general overview we especially invite everyone who has experiences in this field and of course
    everyone who is interested in the topic.

    The AK will be held in english. For the protocol we will make a german summary.

    Preperation:
    In case that you have such an event at your university, we are happy to get some materials (slides, handouts,...)
    or a list of questions the students ask you (as student representatives).

  \subsection*{Protokoll}
    Organisatorial things: suggestion to sit next to someone who can translate if there are difficulties with understanding English.

    Outline:
    \begin{itemize}
      \item Introduction
      \item Situation at other Universities
      \item Talk about Visa and Rathaus
      \item Create a handout to plan a welcome at other universities
    \end{itemize}

    Introduction of Lina who has made experiences being in other countries and wanted to improve things in Siegen.
    Introduction of Siddhu from India who is studying in Siegen now and has experienced problems that he would like to help new international students.

    We will mostly talk about the situation in Germany, but it can also be applied in Austria.
    The situation even varies from university to university.

    Situation in other universities: Do you have a program in the first weeks for welcoming new students?
    About 5/6 universities say yes.
    Do you have international students and where do they come from?
    Austria: mostly Germans. München: students from all over the world, not just Europe.

    Information on what is being done:
    \begin{itemize}
      \item Marburg and Darmstadt: international students are all introduced at university in general, not Physics specifically- only vague information for students.
      \item München: Bachelor studies in German, only Masters in English
        \begin{itemize}
          \item they are shown the city and talked through the study.
          \item Studierendenwerk helps with accomodation and finances.
          \item Also program from international office: peer-to-peer mentoring: an international student is shown around by a German student.
        \end{itemize}
      \item Oldenburg: introduction week mostly in German, but introduction to online-system of the university and a breakfast with time table assistance specifically in English for international students.
      \item Ilmenau: GEZ and paperwork are important because they are mostly in German
      \item Oldenburg: not all of the international students participate in the orientation week, the contact is often lost afterwards.
      \item HU Berlin: Erasmus students are told about the Studienordnung because the English translation is hard to find and learning agreements or getting credits transferred are important to international students.
      \item Marburg: a lot of internet pages have not been translated
      \item Oldenburg: Studienordnung was tried to change, however it is a legal document- cannot be translated easily, universities don't want to have it translated to avoid legal complications -> is there another way?
      \item Würzburg: universities just don't want to translate the documents.
      \item Gießen: even if there aren't official translations, students could provide private translations in order to help international students.
      \item HU Berlin: connections between German and international students are not as close as they could be
      \item Oldenburg: a lot of work in basic lab is done in two. Often, these pairs are made up of one German and one international student.
      \item LMU München: most documents are in English too because other universities have it too.
    \end{itemize}

    A document was prepared to use as a base of discussion during the AK\footnote{\url{https://zapf.wiki/images/3/36/AK_Organising_an_International_Welcome.pdf}}
    How can it be improved?

    There are two different registrational processes to be done by visiting students:
    \begin{itemize}
      \item Visas: often need to be extended during the stay in Germany, so it is crucial for students while they are here. It is important for international students to know the criteria used to decide whether the extension (normally for 3-6 months up to 2 years) ist declared. One of them is: you have to have a good enough financial situation to keep studying
      \item Another one is that the local registration at every cities own local register (Rathaus). You have to inform the local authorities as early as possible that you are living in the city- can be as little as 14 days. This is true even if they don't have a permanent residence yet (for example if they are living in a hotel). Once a permanent residence is found, the Rathaus needs to be updated.
      \item IMPORTANT: Registration and Visa-Extension are not the same thing!
    \end{itemize}

    Opinions:
    \begin{itemize}
      \item Marburg: Thinks document is good, but doubts that a single document can be used in every city. Some of the pieces of information are too specific. In addition, the students often get a lot of information from the international office.
      \item Siddhu: Two sections: university-independent (general information) and university-dependent (information which differ from university by university).
      \item München: all information for every university should be collected so the universities can delete things that don't apply to them. General points: What is needed? Examples: in Munich the Kreisverwaltungsreferat is responsible for administration instead of the Rathaus. In addition, there are specific rules about the usage of the student ticket. She believes that the text should stay like this and everyone can add or delete the relevant information. The specific informations should be clearly marked.
      \item Alumni: keep in mind that redundancy needs to be avoided- only a short introduction to the Fachschaft and the university, not too extensive.
      \item Lina: Now, we should start to talk about specific improvements
      \item München: We should add links to international offices for the university.
      \item HU Berlin: good idea to seperate general and specific information.
      \item Lina: add notice to top that each university should personalize the document.
      \item Alumni: Kid support - people don't need to be married to have kids. More specific information what kind of support exists for students with kids, including links.
      \item HU Berlin: are there international students with kids?
      \item Lina: yes.
      \item Alumni: Facebook pages should not be in the general document.
      \item Gießen: If we want to help students, this is another way for them to get information and we should tell them about it.
      \item Siddhu: He personally found it helpful to use Facebook as a way of connecting with others who speak English, for example for finding accomodation.
      \item München: change to "Social Media and Websites" to include other social media and websites.
      \item HU Berlin: Page 6: Deutsche Bahn is private- the other services should not be called private as a contrast. Page 3: evreything about the student ticket should be put in the "Specific"- section.
      \item Oldenburg: The semester ticket is not always mandatory.
    \end{itemize}

    From now on the proposed paper will be dealt with from the start
      \begin{itemize}
        \item Accomodation: include student housing
          \begin{itemize}
            \item not just wg-gesucht but Google in general or declare those sites explicitely as examples
            \item Münster: there is a program "eine Couch für Ersties" which offers temporary accomodation for new students.
            \item HU Berlin: what are youtube links?
            \item Siddhu: A student explains everything about studying abroad in Germany, could be very useful -> write that in the document.
          \end{itemize}
        \item Broadcasting fees:
          \begin{itemize}
            \item in Austria: only if it is proven, that you have a TV or radio $\rightarrow$ if you do not let controlling persons in, you will not need to pay
            \item GEZ is paid in household, not house.
            \item HU Berlin: BAföG exempts you from paying GEZ. Is there a similar option for international students? Should be checked out and added to the document.
          \end{itemize}
        \item Bus ticket:
          \begin{itemize}
            \item HU Berlin: it is confusing that the semester ticket is mentioned before it is explained later.
            \item München: information about the period in which the semester ticket is valid should be added.
            \item should be changed to "public transport ticket" instead of "bus ticket"
            \item Uni Saarland: shouldn't students know about bus tickets already?
            \item Siddhu: no, it is sometimes not clear how to use day tickets etc.
            \item Wien: information about public transformation is very important to save money.
            \item Würzburg: monthly tickets should be mentioned as well, should be ordered in a way that is easily understandable.
            \item Munich: should be marked to be changed in each city.
            \item Wien: Biking should be added. Bikes are often cheaper and make you feel more integrated into the student life. Add second-hand-shops as well as systems to borrow a bike etc. and the information that many cities have a bike-renting system that is included in the public transport system.
            \item Wien: it should be made clear that this is in chronological order.
            \item Oldenburg: this is mostly information for students before they get enrolled. how do universities get in touch with international students before they are in the university?
            \item Wien: sometimes they send mails before coming to ask for help.
            \item Gießen: Previous contact exists through exchange programs
            \item Dresden: There is no contact with international students, it is difficult to find them and stay in touch
            \item Wien: is there a difference between Germany and other countries in how to write formal mails? Example: use titles and last names etc.
            \item Würzburg: behaviorial code of conduct- maybe include a link or write a separate document.
            \item Munich: some universities in Germany also use the first names of Professors- it is different from university to university.
          \end{itemize}

        \item Rathaus:
          \begin{itemize}
            \item both deadline and the name of registration office are regionally specific. In addition, a paper from the landlord (?) is needed for regristration.
            \item Vienna: deadline is two days.
            \item Würzburg: in some regeristratoin offices it is possible to get an appointment online.
            \item Uni Saarland: sometimes an appointment is necessary, sometimes not possible. This information should be added.
            \item Köln: there is a general telephone number, 115, that connects you to the local Rathaus, but is only implemented in some regions\footnote{\url{https://www.115.de/DE/Startseite/startseite_node.html} has a map. However, the language is German}.
          \end{itemize}

        \item Health insurance
          \begin{itemize}
            \item Würzburg: ticks are only a relevant risk in specific regions in Germany- section could go in the specific part.
            \item Wien: add "how to go to a doctor"- section should be added because the system varies.
            \item Würzburg: this is also different in the regions of Germany.
            \item Oldenburg: make sure it is obvious that visiting the "Hausarzt" is not necessary in emergency.
            \item HU Berlin: this is not necessary, people will know when there is a real emergency
            \item Würzburg: add sentence about emergency room
            \item Gießen: if you are not sure, call 112, they can tell you where to go
            \item Saarland: add sentence about what to do if you are sick on the weekend.
            \item HU Berlin: add sentence that it is not a bad thing to go to a doctor because it is
            \item Gießen: Telephone number for the 'Ärztlicher Bereitschaftsdienst': 116 117 for minor but serious incidents
          \end{itemize}
        \item Liability insurance:
          \begin{itemize}
            \item Oldenburg: losing keys is often not covered
          \end{itemize}
        \item Bank account:
          \begin{itemize}
            \item Munich: update the minimum amount in the account each year
          \end{itemize}
        \item Visa extension: no comment for improvement
        \item Semester ticket: not general: second and third paragraph marked for checking
        \item Student Managment System
          \begin{itemize}
            \item entire section should be checked
            \item sometimes several systems
          \end{itemize}

        Order should be changed: WIFI before System explanatoin
        \item Vienna: add section about how to get a phone/ get a German phone number
        \item University website:
          \begin{itemize}
            \item Würzburg: "spam website" instead of "illegal website"
            \item Darmstadt: eduroam is acessible all over the world. Also, there should be a tutorial about how to register on eduroam
            \item Gießen: and don't forget to use the complete username like username@domain.de
            \item München: eduroam has official rules, link should be added. $\rightarrow$ we can't find this rules
          \end{itemize}
        \item University Email
          \begin{itemize}
            \item Würzburg: link to tutorial about how to redirect emails to private account
          \end{itemize}

        \item Lectures:
          \begin{itemize}
            \item Würzburg: are examples necessary? Siddhu: yes, they can be helpful to know what to ask for.
            \item Wien: mark this one for change as well.
            \item Gießen: mention that eduroam is an open wifi
          \end{itemize}

        \item Timings
          \begin{itemize}
            \item Würzburg: wording of the explanation of s.t. is confusing
            \item Gießen: seldomly used: ,magna com tempore' -> half hour later (very seldomly used)
            \item Wien: Mention that Germans are rather punctual so be there on time or, even better, 5 minutes earlier
          \end{itemize}

        \item Master Thesis
          \begin{itemize}
            \item mark everything for change
            \item Würzburg: why is Master thesis in there? Why not Staatsexam or Bachelor thesis?
            \item Lina: we mostly know master students.
            \item Würzburg/ München: some of the international students aren't master students.
            \item Wien: add how to find a supervisor for thesis (bachelor and master)
          \end{itemize}

        \item Holidays
          \begin{itemize}
            \item Marburg: most of the shops are closed on holidays
            \item Oldenburg: each university should add their holidays
            \item Wien: link that is updated each year
            \item Würzburg: say that semester holidays are not always free
            \item München: add local opening hours
            \item Göttingen: shops are closed on Sundays -> make separate passage about shopping
          \end{itemize}

        \item Mensa
          \begin{itemize}
            \item local, mark for change
            \item basic points: opening hours, prices, how to pay, how to find ingredients (allergies), where it is...
          \end{itemize}

        \item maximum time for course
          \begin{itemize}
            \item local, mark for change
            \item München: include sentence that the number of semesters can be extended in certain cases such as illness and pregnancy
            \item Lina: keep somthing about the visa in there
          \end{itemize}

        \item student representatives
          \begin{itemize}
            \item München: add that you are welcome to take part in it
            \item mark for change
          \end{itemize}

        \item additional courses
          \begin{itemize}
            \item mark for change
          \end{itemize}
      \end{itemize}

    Suggestions for things to add:
      \begin{itemize}
        \item Add something about time tables (marked for change)
        \item HU Berlin: leave space for other institutions of the university
        \item München: how to sign up for exams
        \item Darmstadt: add infomation about orientation weeks
        \item Würzburg: link to informational office
        \item Darmstadt: Fachschaften should check which documents are available in German
        \item Lina: add list of documents that are not availabe in english and add suggestions where to get the information
        \item include section about Studentenwerk what it is and what it does (sometimes called differently)
        \item Mentoring programs/ peer-to-peer mentoring if availabe
        \item check how international and German students can be connected
      \end{itemize}

    Discussion about how to distribute document: word document or (preferably) plain text. \\

    A follow-up AK is suggested for Würzburg to discuss problems we have found more deeply.
    Munich: a comment section should be added to the protocol to allow for further suggestions.
