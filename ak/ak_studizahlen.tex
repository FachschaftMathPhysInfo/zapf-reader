% !TEX TS-program = pdflatex
% !TEX encoding = UTF-8 Unicode
% !TEX ROOT = main.tex

\section{AK Rückläufige Studierendenzahlen}

	\textbf{Protokoll vom:} 31.05.2018,
	Beginn: 16:30 Uhr,
	Ende: 18:30 Uhr \\
	\textbf{Redeleitung:} Peter Steinmüller (KIT) \\
	\textbf{Protokoll:} Peter Steinmüller (KIT) \\
	\textbf{anwesende Fachschaften:} Universität Augsburg, Universität Bochum Technische Universität Chemnitz, Brandenburgische Technische Universität Cottbus, Technische Universität Darmstadt, Technische Universität Dresden, Technische Universität Bergakademie Freiberg, Technische Universität Graz, Technische Universität Ilmenau, Friedrich-Schiller-Universität Jena, Universität Konstanz, Philipps-Universität Marburg, Carl von Ossietzky Universität Oldenburg, Universität des Saarlandes, Karlsruher Institut für Technologie, Julius-Maximilians-Universität Würzburg,

	\subsection*{Informationen zum AK}
		\begin{itemize}
			\item \textbf{Ziel des AKs}: An einigen Unis gibt es kleinere Studiengänge, wie etwa Geo-Physik oder Meteorologie. Die Frage ist, wie Studenten in diese Studiengänge gebracht werden können, um die Studiengänge zu erhalten.
			% \item \textbf{Folge-AK}:
			% \item \textbf{Vorwissen}:
      % \item \textbf{Materialien}:
			% \item \textbf{Zielgruppe}:
			% \item \textbf{Ablauf}:
			% \item \textbf{Voraussetzungen}:
		\end{itemize}

  \subsection*{Protokoll}
    \paragraph{Erfahrungen zu kleinen physiknahen Studiengängen}
      \begin{itemize}
        \item Cottbus: Seit Jahren im "puren" Physik-Studiengang sehr geringe Studierendenzahl. \underline{Gesamtanzahl} (1. Semester - Abschluss) O(40).
        \item Marburg: Die Studierendenzahlen steigen schon seit langem nicht an. Der Fakultät ist es erst letztes Jahr aufgefallen. Mit der Hilfe der Professoren sollen jetzt Maßnahmen getroffen werden. Angebot eines Notfalltutoriums aus Qualitätssicherungsmitteln, von Studierenden und Professoren wurde diese Idee als gut befunden.
        \item Augsburg: Studiengang Materialwissenschaften wird eingestampft; stattdessen soll mit dem Studiengang Wirtschaftsingenieurwesen kooperiert und ein neuer Studiengang mit anderer Ausrichtung geschaffen werden. Die Fachschaft versucht lokal die Abiturenten direkt/während/nach dem Abitur abzuholen.
        \item Ilmenau: Die Zahlen von Technischer Physik sind in etwa konstant; ein Professor bietet für Schüler Projekte an; Der Studiengang verliert immer wieder Studenten an Ingenieursstudiengänge, es wechseln allerdings kaum Studenten in den Studiengang; Erstis werden möglichst schnell in die Studierendengemeinschaft aufgenommen und eingebunden.
        \item Karlsruhe: Es werden speziell für Erstsemester zusätzliche Mentoren ausgebildet, um die Studieneingangsphase zu erleichtern.
        \item Bochum: Erläutert Problematik über die falsche Einschätzung der Erstsemester bezüglich des Faches, die dazu führt, dass Studierende schnell aufhören Physik zu studieren.
        \item Freiberg: Fachschaft versucht den Kontakt zu den Studierenden zu knüpfen, bzw. als Ansprechpartner erkenntlich zu machen und die Studierenden auf die Prüfungen vorzubereiten
        \item Jena: Klausurvorbereitungswochenende, bei dem Übungsblätter und Altklausuren besprochen werden; hat sich sehr positiv auf die Durchfallsquoten ausgewirkt; Finanzierung über geringen Teilnehmerbeitrag
        \item Graz: Die ersten Übungsblätter werden mit den Studienanfängern gemeinsam gerechnet, daraus entstehen dann Übungsgruppen. Grobe Einschätzung über den Grund der Abbrecher: Die Studierenden haben sich was anderes vorgstellt.
        \item Chemnitz: hat die Vermutung, dass die Studierenden nicht auf ihren Ort aufmerksam wird.
      \end{itemize}

		\paragraph{Leitfragen}
			\begin{itemize}
				\item Wollen Leute nicht in den Ort?
				\item Sind die Prüfungen zu schwer?
				\item Ist das Studium nicht das richtige?
			\end{itemize}

		\paragraph{Ideen gegen Abbrecher}
			\begin{itemize}
				\item Spezielle Tutorien, nach dem Motto Notfall- oder Beratungstutorium
				\item Aufstellen und Ausbilden von Mentoren
				\item Fachschaft aktiv als Berater und Ansprechpartner bewerben
			\end{itemize}

		\paragraph{Werbung für kleinere Studiengänge}
			\begin{outline}
				\1 Karlsruhe:
				  \2 Es wurden StudienbotschafterInnen eingerichtet, die in die umliegenden Gymnasium gehen und aktiv Werbung für ihre Studiengänge machen.
				  \2 jDPG macht im Prinzip sowas ähnliches.
				  \2 Das Schülerlabor bietet das verfügbare Equipment zur Werbung an.
				  \2 Schülerwerbung auch über den ``Uni für Einsteiger'' Tag. Dabei sollen an der Universität kleine Aufbauten organisiert werden mit kleinen Ständen, vor allem Experimenten, und Studierenden, die etwas aus ihrem Studium berichten können.
				\1 Graz: Es gibt für die Schulabsolventen einen extra Beratungstermin mit Studierenden.
				\1 Cottbus: Die Uni bietet einige Aktivitäten an, jedoch wird das von der Stadt nicht wirklich beworben. \textit{Hinweis:} Erstsitz Initiative als Vorteil für die Stadt.
				\1 Würzburg:
				  \2 Große Zusammenarbeit mit der Stadt, die diverse Vorteile für die Studierenden bieten kann.
				  \2 Man wendet sich nicht nur an die Schulen und Universitäten in der näheren Umgebung, sondern auch in größeren Umfeld (etwa Frankfurt)
				\1 \textbf{Allgemein:}
					\2 Studiengänge ins CHE Ranking oder in vergleichbare Übersichten bekommen.
					\2 Uni-/Fachschafts-Seiten sollen kompakt und übersichtlich gestalltet werden. (etwa \url{http://fachschaft.physik.kit.edu/drupal/})
					\2 Erstellung und Überarbeitung des Studienführers.

				\1 \textbf{Frage}: Ist es möglich über die jDPG/DPG Sachen, wie den Studienführer zu verteilen? \\
				  $\rightarrow$ Man könnte der DPG einen bereits fertigen Flyer etwa über den Studienführer geben.
			\end{outline}

		\paragraph{Ideen}
			\begin{outline}
				\1 An den Schulen gezielter Werbung für die kleinen Studiengänge machen, da Abiturienten eventuell gar nicht wissen, dass es die speziellen Studiengänge gibt.
			  \1 Kleine Studiengänge aktiv in den Studienführer aufnehmen, damit diese gleichberechtigt zu den großen Universitäten und Studiengängen beworben werden.
			  \1 Würzburg: Man kann eine Karte erstellen, an welchen Orten überall Physik studierbar ist. Dabei soll bei dem Ortsnamen dabei stehen, welche physiknahen Studiengänge möglich sind.
			  \1 Würzburg: Die Studierenden in den kleinen Fächern aktiv befragen, warum sie diese Wahl getroffen haben. Möglicherweise könnte man so eine gezieltere Zielgruppe herausfinden.
			  \1 Studie am Studienbeginn zur Frage, warum es zur Wahl des Studiengangs und des -Ortes kam.
			      \2 Könnte man über die Vorkurse und O-Phase verteilen.
			  \1 Schüler motivieren die Stadt zu wechseln (fraglich).
			\end{outline}

		\paragraph{Fragen}
			\begin{itemize}
				\item Wie sehen die Studierendenzahlen allgemein aus? Gehen diese auch bei Physik allgemein zurück?
			  \item Wie könnt ihr an eurer Uni/in eurer Stadt Studienzahlen verbesseren?
			  \item Wollen einfach die Abiturienten generell weniger studieren? Physik mit anderen Studiengängen vergleichen.
			\end{itemize}

		\textbf{An die MeTaFa tragen!}
