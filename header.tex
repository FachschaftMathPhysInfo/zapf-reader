% !TEX ROOT = main.tex

\usepackage{afterpage}
\usepackage{amsmath}
\usepackage[ngerman]{babel}
\usepackage{blindtext}
\usepackage{csquotes}
% \usepackage{enumitem} % error: underfull hbox
\usepackage{eurosym}
\usepackage{fancyhdr}
\usepackage{float}
\usepackage{fontawesome}
\usepackage[T1]{fontenc}
\usepackage{geometry}
\usepackage[pdftex]{graphicx}
%\usepackage[scaled=0.81]{helvet}     % Schriftart fuer Resos
\usepackage{hyperref}
\usepackage[utf8]{inputenc}
\usepackage{libertine}
\usepackage{lmodern}     % Ersatz fuer Computer Modern-Schriften
\usepackage{mdwlist}     % Änderung der Zeilenabstände bei itemize und enumerate
\usepackage{microtype}
\usepackage{minitoc}
\usepackage{outlines}
\usepackage{paralist}
\usepackage{pdfpages}
\usepackage{subcaption}
\usepackage{tcolorbox}
\usepackage{tocloft}
\usepackage{todonotes}
\usepackage{ulem}
\usepackage{url}
\usepackage{wrapfig}
\usepackage{xcolor}
% \usepackage[draft, markup=underlined]{changes}

  % braucht man das?
%\usepackage{titlesec}    % Abstand nach Überschriften neu definieren
%\titlespacing{\subsection}{0ex}{3ex}{-1ex}
%\titlespacing{\subsubsection}{0ex}{3ex}{-1ex}

\definecolor{urlred}{HTML}{660000}
\hypersetup{ % Design der Hyperrefs
    colorlinks=true,
    linkcolor=black,    % Farbe der internen Links (u.a. Table of Contents)
    urlcolor=black,    % Farbe der url-links
    citecolor=black} % Farbe der Literaturverzeichnis-Links

\parindent 0pt                 % Absatzeinrücken verhindern (pck:blindtext)
\parskip 4pt                 % Absätze durch Lücke trennen

% Suche nach Grafiken in ./media und .:
% \graphicspath{{./media/}{./}}

% Satzspiegel
\geometry{papersize={154mm,216mm}, layout=a5paper, layouthoffset=3mm, layoutvoffset=3mm, inner=20mm, outer=15mm, top=15mm, bottom=15mm, heightrounded, marginparwidth=7mm, marginparsep=5mm}
\setlength{\parindent}{0pt}

% Schriftarten
\fontfamily{pag}\selectfont % Avantgar
%\renewcommand{\sfdefault}{phv}
%\newenvironment{resofont}{\fontfamily{phv}\selectfont}{\par}

% itemize item titlespacing
% ??? ohne enumitem

% Headlines
\pagestyle{fancy}
\fancyhf{}
\definecolor{unicolor}{HTML}{B5152B}
\renewcommand{\headrulewidth}{1.5pt} % Strich unterm Header
\renewcommand{\footrulewidth}{0pt} % Strich überm Footer
\renewcommand{\chaptermark}[1]{ \markboth{\quad \quad \quad #1}{}} % chapter Titel mit "Formatierung"
\renewcommand{\sectionmark}[1]{ \markright{#1 \quad \quad \quad}{}} % section Titel mit "Formatierung"
\newlength\FHoffset
\setlength\FHoffset{1cm}
\addtolength\headwidth{1.5\FHoffset} % andere Länge für Header als Text
\fancyheadoffset{\FHoffset}
\fancyfootoffset{\FHoffset}

\renewcommand{\headrule}{\hbox to\headwidth{%
  \color{unicolor}\leaders\hrule height \headrulewidth\hfill}}
% \renewcommand{\footrule}{\hbox to\headwidth{%
%   \color{unicolor}\leaders\hrule height \headrulewidth\hfill}}

\lhead{\leftmark}
\chead{}
\rhead{\rightmark}
\lfoot{}
\cfoot{
  \begin{tikzpicture}
    \node at (-0.5, 0) {\includegraphics[width=1cm, height=1cm]{vorlagen/logo2.png}};
    \draw (0,-0.5) -- (0,0.5);
    \node at (0.5,0) {\thepage};
  \end{tikzpicture}
}
\rfoot{}

% Boxen etwa für Abstimmungen in Plenen
\newenvironment{success}[1]{
  \begin{center}
    \begin{tcolorbox}[colback=green!5!white,colframe=green!60!black, title=#1]
    }
    {
    \end{tcolorbox}
  \end{center}
}

\newenvironment{info}[1]{
  \begin{center}
    \begin{tcolorbox}[colback=blue!5!white,colframe=blue!60!black, title=#1]
    }
    {
    \end{tcolorbox}
  \end{center}
}

\newenvironment{danger}[1]{
  \begin{center}
    \begin{tcolorbox}[colback=yellow!5!white,colframe=yellow!75!black, title=#1]
    }
    {
    \end{tcolorbox}
  \end{center}
}

% used for resos and pospaps
% \pagestyle{empty}
% \cfoot{}
% \lfoot{Zusammenkunft aller Physik-Fachschaften}
% \rfoot{www.zapfev.de\\stapf@zapf.in}
% \renewcommand{\headrulewidth}{0pt}
% \renewcommand{\footrulewidth}{0.1pt}
% \newcommand{\gen}{*innen}

%Unterdrückt Section und Subsection Nummern, also werden nur Chapter Nummer angezeigt
\setcounter{secnumdepth}{0}
